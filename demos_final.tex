\documentclass[12pt,a4paper]{article}
\sloppy
\usepackage[utf8]{inputenc}
\usepackage[english]{babel}
\usepackage[T1]{fontenc}
\usepackage{natbib}
\usepackage{amsmath}
\usepackage{lscape}
\usepackage{amsfonts}
\usepackage{amssymb}
\usepackage{graphicx}
\usepackage[left=2.5cm,right=2.5cm,top=2.5cm,bottom=2.5cm]{geometry}
\usepackage{cancel}
\usepackage{tikz}
\usepackage{float}
\usepackage{multirow}
\usepackage[breaklinks=true]{hyperref}
\usepackage{framed}
\usepackage{listings}
\usepackage{algorithm}
\usepackage[noend]{algpseudocode}

%%%%%%%%%%%%%%%% Informacion de documento %%%%%%%%%%%%%%%%%%%%
% \author{Mansilla, Kevin Gaston\footnote{\href{mailto:kevingston47@gmail.com}{kevingston47@gmail.com}}}
% \title{Discreta II: Demostraciones}
% \date{\today}
%%%%%%%%%%%%%%%%%%%%%%%%%%%%%%%%%%%%%%%%%%%%%%%%%%%%%%%%%%%%%%

\renewcommand{\algorithmicrequire}{\textbf{Input:}}
\renewcommand{\algorithmicensure}{\textbf{Output:}}
\newcommand{\R}{\mathbb{R}}
\newcommand{\N}{\mathbb{N}}
\newcommand{\Q}{\mathbb{Q}}
\newcommand{\Z}{\mathbb{Z}}
\newcommand{\K}{\mathbb{K}}
\newcommand{\<}{\leqslant}
\renewcommand{\>}{\geqslant}
\providecommand{\abs}[1]{\left|#1\right|}
\providecommand{\norma}[1]{\left||#1\right||}
\providecommand{\indexa}{\triangleright}
\providecommand{\parent}[1]{\left(#1\right)}
\providecommand{\corchet}[1]{\left[#1\right]}
\providecommand{\prop}[1]{\langle #1\rangle}
\providecommand{\llaves}[1]{\left\lbrace #1 \right\rbrace}
\newtheorem{teorema}{Teorema}
\newtheorem{corolario}{Corolario}
\newtheorem{definition}{Definición}
\newtheorem{ejemplo}{Ejemplo}
\newtheorem{nota}{Nota}
\newtheorem{lema}{Lemma}
\newtheorem{ejercicio}{Ejercicio}
\renewcommand{\tablename}{Tabla}
\renewcommand\refname{Bibliograf{\'i}a}
\renewcommand{\listtablename}{Índice de tablas}
\renewcommand{\figurename}{Gráfico}
\renewcommand{\thesection}{\Roman{section}}

%%
%\documentclass[12pt,a4paper]{article}
%\sloppy
\usepackage[utf8]{inputenc}
\usepackage[english]{babel}
\usepackage[T1]{fontenc}
\usepackage{natbib}
\usepackage{amsmath}
\usepackage{lscape}
\usepackage{amsfonts}
\usepackage{amssymb}
\usepackage{graphicx}
\usepackage[left=2.5cm,right=2.5cm,top=2.5cm,bottom=2.5cm]{geometry}
\usepackage{cancel}
\usepackage{tikz}
\usepackage{float}
\usepackage{multirow}
\usepackage[breaklinks=true]{hyperref}
\usepackage{framed}
\usepackage{listings}
\usepackage{algorithm}
\usepackage[noend]{algpseudocode}

%%%%%%%%%%%%%%%% Informacion de documento %%%%%%%%%%%%%%%%%%%%
% \author{Mansilla, Kevin Gaston\footnote{\href{mailto:kevingston47@gmail.com}{kevingston47@gmail.com}}}
% \title{Discreta II: Demostraciones}
% \date{\today}
%%%%%%%%%%%%%%%%%%%%%%%%%%%%%%%%%%%%%%%%%%%%%%%%%%%%%%%%%%%%%%

\renewcommand{\algorithmicrequire}{\textbf{Input:}}
\renewcommand{\algorithmicensure}{\textbf{Output:}}
\newcommand{\R}{\mathbb{R}}
\newcommand{\N}{\mathbb{N}}
\newcommand{\Q}{\mathbb{Q}}
\newcommand{\Z}{\mathbb{Z}}
\newcommand{\K}{\mathbb{K}}
\newcommand{\<}{\leqslant}
\renewcommand{\>}{\geqslant}
\providecommand{\abs}[1]{\left|#1\right|}
\providecommand{\norma}[1]{\left||#1\right||}
\providecommand{\indexa}{\triangleright}
\providecommand{\parent}[1]{\left(#1\right)}
\providecommand{\corchet}[1]{\left[#1\right]}
\providecommand{\prop}[1]{\langle #1\rangle}
\providecommand{\llaves}[1]{\left\lbrace #1 \right\rbrace}
\newtheorem{teorema}{Teorema}
\newtheorem{corolario}{Corolario}
\newtheorem{definition}{Definición}
\newtheorem{ejemplo}{Ejemplo}
\newtheorem{nota}{Nota}
\newtheorem{lema}{Lemma}
\newtheorem{ejercicio}{Ejercicio}
\renewcommand{\tablename}{Tabla}
\renewcommand\refname{Bibliograf{\'i}a}
\renewcommand{\listtablename}{Índice de tablas}
\renewcommand{\figurename}{Gráfico}
\renewcommand{\thesection}{\Roman{section}}

%%
%\documentclass[12pt,a4paper]{article}
%\sloppy
\usepackage[utf8]{inputenc}
\usepackage[english]{babel}
\usepackage[T1]{fontenc}
\usepackage{natbib}
\usepackage{amsmath}
\usepackage{lscape}
\usepackage{amsfonts}
\usepackage{amssymb}
\usepackage{graphicx}
\usepackage[left=2.5cm,right=2.5cm,top=2.5cm,bottom=2.5cm]{geometry}
\usepackage{cancel}
\usepackage{tikz}
\usepackage{float}
\usepackage{multirow}
\usepackage[breaklinks=true]{hyperref}
\usepackage{framed}
\usepackage{listings}
\usepackage{algorithm}
\usepackage[noend]{algpseudocode}

%%%%%%%%%%%%%%%% Informacion de documento %%%%%%%%%%%%%%%%%%%%
% \author{Mansilla, Kevin Gaston\footnote{\href{mailto:kevingston47@gmail.com}{kevingston47@gmail.com}}}
% \title{Discreta II: Demostraciones}
% \date{\today}
%%%%%%%%%%%%%%%%%%%%%%%%%%%%%%%%%%%%%%%%%%%%%%%%%%%%%%%%%%%%%%

\renewcommand{\algorithmicrequire}{\textbf{Input:}}
\renewcommand{\algorithmicensure}{\textbf{Output:}}
\newcommand{\R}{\mathbb{R}}
\newcommand{\N}{\mathbb{N}}
\newcommand{\Q}{\mathbb{Q}}
\newcommand{\Z}{\mathbb{Z}}
\newcommand{\K}{\mathbb{K}}
\newcommand{\<}{\leqslant}
\renewcommand{\>}{\geqslant}
\providecommand{\abs}[1]{\left|#1\right|}
\providecommand{\norma}[1]{\left||#1\right||}
\providecommand{\indexa}{\triangleright}
\providecommand{\parent}[1]{\left(#1\right)}
\providecommand{\corchet}[1]{\left[#1\right]}
\providecommand{\prop}[1]{\langle #1\rangle}
\providecommand{\llaves}[1]{\left\lbrace #1 \right\rbrace}
\newtheorem{teorema}{Teorema}
\newtheorem{corolario}{Corolario}
\newtheorem{definition}{Definición}
\newtheorem{ejemplo}{Ejemplo}
\newtheorem{nota}{Nota}
\newtheorem{lema}{Lemma}
\newtheorem{ejercicio}{Ejercicio}
\renewcommand{\tablename}{Tabla}
\renewcommand\refname{Bibliograf{\'i}a}
\renewcommand{\listtablename}{Índice de tablas}
\renewcommand{\figurename}{Gráfico}
\renewcommand{\thesection}{\Roman{section}}

%%
%\documentclass[12pt,a4paper]{article}
%\input{packages.tex}
% \begin{document}
%\maketitle{}


% \end{document}

% \begin{document}
%\maketitle{}


% \end{document}

% \begin{document}
%\maketitle{}


% \end{document}

\author{Mansilla, Kevin Gaston\footnote{\href{mailto:kevingston47@gmail.com}{kevingston47@gmail.com}}}
\title{Discreta II: Demos final}
\date{\today}
\begin{document}
\maketitle{}

%%%%%% 1 %%%%%%
\begin{itemize}
    \item [1)] \textbf{Cual es la complejidad del algoritmo de Edmonds-Karp? Probarlo (
        Nota: en la prueba se definen unas distancias, y se prueba que esas 
        distancias no disminuyen en pasos sucesivos de EK. Ud. puede usar esto 
        sin necesidad de probarlo.)}
        \label{dem:EK}
\end{itemize}

Diremos que un lado se vuelve critico al pasar de $f_{k}$ a $f_{k+1}$ si 
se \textbf{satura} o \textbf{vacia}. Entonces cuántas veces puede un lado volverse 
crítico?
\medskip

Supongamos que un lado se vuelve critico en el paso $k$ y luego en el paso $j$, 
donde $k < j$. Entonces:
$$s \ldots \underbrace{xz}_{critico}\ldots t$$
pasa de $f_{k}$ a $f_{k+1}$.
\medskip

Si $\overrightarrow{xz} \in E$ se satura tenemos que $d_{k}(z) = d_{k}(x)+1$, para 
volverse critico otra vez, o bien se vacia o bien se vuelve a saturar, pero para 
esto último primero tiene que devolver flujo.
\medskip

En cualquier caso, $\exists l: k \< l \< j$ en donde devolvio flujo, es decir, tal 
que para pasar de $f_{l}$ a $f_{l+1}$, devolvimos flujo (lado backward), como se 
muestra en el siguiente camino:
$$s \ldots \overleftarrow{zx} \ldots t, f_{l} \to f_{l+1}$$

por EK sabemos que $d_{l}(x) = d_{l}(z) + 1$ (distancias no disminuyen).
\medskip

Si lo que tenemos es $\overleftarrow{zx} \in E$ (backward), con un camino 
$s \ldots \overleftarrow{zx} \ldots t \Rightarrow d_{k}(x) = d_{k}(z) +1$ el 
analisis es similar, se volvio critico porque se vacio, entonces para volver a 
ser critico debe saturarse o volverse a vaciar, para esto último debe haber 
enviado flujo.
\medskip

Entonces, $\exists l: k \< l \< j$ para pasar de $f_{l}$ a $f_{l+1}$, enviamos 
flujo, como se muestra en el siguiente camino:
$$s\ldots \overrightarrow{xz} \ldots t, f_{l} \to f_{l+1}$$

Entonces, por EK sabemos: $d_{l}(z) = d_{l}(x) + 1$.
\medskip

En cualquier caso, tenemos en terminos de $f_{k}$ y $f_{k+1}$
\begin{align*}
    d_{k}(z) &= d_{k}(x) +1 \,\,(\text{fordward})\\
    d_{k}(x) &= d_{k}(z) +1 \,\,(\text{backward})
\end{align*}
Entonces
\begin{align*}
    d_{l}(t) = d_{l}(x) + b_{l}(x) &= d_{l}(z) + 1 + b_{l}(x)\\
    &=\llaves{\text{Como las distancias no disminuyen}}\\
    &\> d_{k}(z) + b_{k}(x) + 1 = d_{k}(x) + 1 + b_{k}(x) +1 = d_{k}(x) + b_{k}(x)+2\\
    &\Rightarrow d_{l}(t) = d_{k}(t) +2
\end{align*}

Conclusión: Una vez que un lado se vuelve crítico solo puede volver a ser crítico 
si la distancia entre $s$ y $t$ aumenta en por lo menos $2$. Un lado puede 
volverse crítico a lo sumo $\frac{n-1}{2}$ veces entonces es $O(n)$.
\medskip

\begin{itemize}
    \item [1.] Para pasar de $f_{k}$ a $f_{k+1}$ al menos un lado se vuelve crítico.
    \item [2.] Hay $m$ lados.
    \item [3.] Cada lado se vuelve critico $O(n)$ veces.\\
        Por lo tanto el número total de pasos es $O(nm)$.\\
        La complejidad de hallar un camino aumentate es $O(m)$ por BFS.\\
        Entonces la complejidad total es $O(m)*O(nm)=O(nm^{2})$.
\end{itemize}

%%%%%% 2 %%%%%%
\begin{itemize}
    \item [2)] \textbf{Probar que si, dados vértices $x$, $z$ y flujo $f$ definimos 
    a la distancia entre $x$ y $z$ relativa a $f$ como la longitud del menor 
    $f-$camino aumentante entre $x$ y $z$, si es que existe tal camino, o infinito 
    si no existe o 0 si $x = z$, denotandola por $d_{f} (x, z)$, y definimos 
    $d_{k}(x) = d_{f_{k}} (s, x)$, donde $f_{k}$ es el $k-$ésimo flujo en una 
    corrida de Edmonds-Karp, entonces $d_{k}(x) \< d_{k+1}(x)$}
    \label{dem:dist}
\end{itemize}

Vamos a demostrar que $d_{k}(x) \< d_{k+1}(x)$.
\medskip

Sea $A = \llaves{y: d_{k+1}(y) < d_{k}(y)}$. Queremos ver que $A = \emptyset$ 
vamos a suponer que no es $\emptyset$ y llegar a un absurdo.
\medskip

Suponemos $A \neq \emptyset$, es decir, que tiene $1$ o muchos elementos, y de 
entre todos los elementos elegimos $x \in A$ tal que:
\begin{equation}
    d_{k+1}(x) \< d_{k+1}(y)\,\, \forall y \in A
    \label{EK:1}
\end{equation}

Como $x \in A$, entonces $d_{k+1}(x) < d_{k}(x) \Rightarrow d_{k+1} < \infty$, por 
definición de $A$.
Entonces, existe $f_{k+1}-$ca entre $s$ y $x$, tomemos uno de menor longitud, es 
decir,
\begin{align*}
    \underbrace{s .... x}_{longitud = d_{k+1}(x)}\\
\end{align*}

\underline{Observación importante:} 
$d_{k}(s) = d_{k+1}(s) = 0, s\notin A \ldots s \neq x$.
\medskip

Sea $z$ el vertice inmediatamente anterior a $x$ en ese camino $(s\ldots zx)$. 
Como el camino es de \textbf{longitud minima}, entonces:
\begin{equation}
    d_{k+1}(x) = d_{k+1}(z) + 1
    \label{EK:2}
\end{equation}

En particular, $d_{k+1}(z) < d_{k+1}(x) \Rightarrow_{Por \eqref{EK:1}} z \notin A$. 
(tendría que ser $\leq$)
\medskip

Como $z \notin A$, entonces se da lo contrario:
\begin{equation}
    d_{k}(z) \< d_{k+1}(z) \Rightarrow d_{k}(z) < \infty
    \label{EK:3}
\end{equation}

Como $d_{k+1}(z) < d_{k+1}(x) < \infty$. Por \eqref{EK:3} existe un $f_{k}-$ca 
entre $s$ y $z$, y de ellos voy a tomar uno de longitud minima.
\medskip

Primero supongamos que $\overrightarrow{zx} \in E$. Entonces, si suponemos 
primero que esta saturado $(f_{k}(\overrightarrow{zx}) = c(\overrightarrow{zx}))$, 
pero $\underbrace{s\ldots z}_{f_{k+1}}x$ es un ca relativo. Entonces, 
$f_{k+1}(\overrightarrow{zx}) < c(\overrightarrow{zx})$. Esto implica que para 
pasar de $f_{k}$ a $f_{k+1}$ el lado se uso backward.
\medskip

Entonces, para pasar de $f_{k}$ a $f_{k+1}$ debe haber un ca de la forma 
$s \ldots \overleftarrow{xz} \ldots t$ y como estamos usando EK este camino es de 
longitud minima.
\begin{equation}
    d_{k}(z) = d_{k}(x) + 1
    \label{EK:4}
\end{equation}

Entonces:
\begin{align*}
    d_{k}(x) =_{\eqref{EK:4}} d_{k}(z)-1 \< _{\eqref{EK:3}} d_{k+1}(z) -1 = (d_{k+1}(x)-1)-1 &= d_{k+1}(x) -2\\
    &<d_{k}(x) -2\\
    &\Rightarrow d_{k}(x) < d_{k}(x)-2 \\
    &\Rightarrow 0 < 2 \,\,\text{Absurdo}
\end{align*}

Llegamos a un absurdo por suponer que $f_{k}(\overrightarrow{zx}) = c(\overrightarrow{zx})$. 
Ahora veamos el otro caso, $f_{k}(\overrightarrow{zx}) < c(\overrightarrow{zx})$ 
(no esta saturado). Pero teniamos el $f-k$ ca, $s \ldots z$ entonces existe un 
$f_{k+1}$ ca del tipo $s\ldots \overrightarrow{zx}$ (no se si es de longitud 
minima). Entonces:
\begin{align*}
    d_{k}(x) &\< d_{k}(z)+1\\
    & \< d_{k+1}(z) +1 = d_{k+1}(x)\\
    & < d_{k}(x) \Rightarrow 0 < 0 \,\,\text{Absurdo}
\end{align*}

Por lo tanto en cualquier caso, si $\overrightarrow{zx} \in E$, llegamos a un 
absurdo.
\medskip

Ahora supongamos que $\overrightarrow{xz} \in E$, en este caso también llegamos 
a un absurdo dependiendo si $f_{k}(\overrightarrow{xz}) = 0$ o no.
\medskip

Entonces, si $f_{k}(\overrightarrow{xz}) = 0$ como $\underbrace{s\ldots z}_{f_{k}}x$ es ca. 
Entonces, $zx$ es backward, $s\ldots \overleftarrow{zx}$, entonces 
$f_{k+1}(\overrightarrow{xz}) > 0$. Para pasar de $f_{k}$ a $f_{k+1}$ usamos un ca, 
$s\ldots \overleftarrow{xz} \ldots t\Rightarrow_{EK} d_{k}(z) = d_{k}(x)+1$ 
y es un absurdo como antes. 
\medskip

Por otro lado, si $f_{k}(\overrightarrow{xz}) > 0$ entonces 
tenemos que $\underbrace{s\ldots \overleftarrow{zx}}_{f_{k+1}}$ es un ca, por lo 
que $d_{k}(x)\< d_{k}(z)+1$ y llegamos a un absurdo otra vez.
\medskip

Entonces hemos llegado a un absurdo en todos los casos, por lo tanto, $A = \emptyset$ 
y $d_{n} < d_{n+1}$ (las distancias no disminuyen).

%%%%%% 3 %%%%%%
\begin{itemize}
    \item [3)] \textbf{Cual es la complejidad del algoritmo de Dinic? Probarla en ambas 
    versiones: Dinitz original y Dinic-Even. (no hace falta probar que la distancia 
    en networks auxiliares sucesivos aumenta)}
    \label{dem:Dinic}
\end{itemize}

\begin{corolario} La complejidad de cualquier algoritmo que use NA, es 
    $$\underbrace{O(n)}_{\text{Num de NA}}\parent{\underbrace{O(m)}_{\text{Contruir NA con BFS}} + \,\text{Complejidad de hallar un FB}}$$
    \label{cor:NA}
\end{corolario}

Por el corolario anterior, basta ver que la complejidad de hallar un 
\textbf{flujo bloqueante} es $O(nm)$.
\medskip

En la versión original, cada camino entre $s$ y $t$ se encuentra usando DFS, 
pero el NA tiene la propiedad extra que se garantiza que toda búsqueda DFS nunca 
va a tener que hacer backtranking, pues cada vértice con lado entrante tiene lado 
saliente. Pero esta propiedad tiene el costo de tener que mantenerla entre 
camino y camino, revisando el NA. A esta operación se la llama \textbf{podar}.
\medskip

Primero veamos la complejidad de encontrar todos los caminos. La construcción de 
cada camino es $O(r)$ donde $r$ es el número de niveles. Como $r<n$ podemos decir 
que esta parte es $O(n)$, luego de cada camino se borran del NA los lados 
saturados, pero esto es simplemente recorrer una vez más el camino, así que 
es $O(n)$
\medskip

Cada camino satura al menos un lado, y luego borramos ese lado (del NA), por lo 
tanto hay a lo sumo $O(\# \,\text{lados del NA}) = O(m)$ caminos.
\medskip

Por lo tanto la complejidad de hallar todos los caminios es $O(n)O(m) = O(nm)$, 
pero falta la complejidad de todos los "Podar", pero cómo funciona podar?
\medskip

Se recorre en los niveles desde $t$ a $s$ mirando cada vertice y sino tiene lados 
lo borramos a él y a todos los lados que llegaban a él.
\medskip

Podar tiene dos partes:
\begin{itemize}
    \item [1.] Revisa los vertices.
    \item [2.] Si es necesario borrar los lados.
\end{itemize}

En $1)$ es para cada vertice, entonces es $O(1)$ y como hay $n$ vertices, el costo 
total de un podar es $O(n)$. Cuántos podar hay? Hay uno por camino y había $O(m)$ 
caminos, por lo tanto hay $O(m)$ podar y la complejidad de revisar los vertices 
es $O(n)$, por lo tanto esta parte también es $O(nm)$.
\medskip

Por otra parte en $2)$ no queremos la complejidad de un podar sino la de todos 
(la de un podar puede ser muy grande y va reduciendo a medida que se borran lados).
Como la segunda parte cada vez que se llama borra al menos un lado, la suma total 
sobre todos los podar es $O(m)$.

$$\text{Complejidad Total} = O(nm) + O(nm) + O(m) = O(nm) \square$$

\underline{Dinitz-Even (Versión Occidental):} La diferencia de la versión 
de Even es que el NA no esta depurado y por lo tanto un DFS puede demorar mas de 
$O(n)$, pero va depurando a medida que se corre DFS, borrando los lados por los 
cuales tenemos que retroceder.
\newpage
\begin{algorithm}
    \caption{Calculo del flujo bloqueante del NA}
    \begin{algorithmic}[1]
        % ENTRADA / SALIDA
        % \Require{$\Gamma^{+}(x)$, NA} 
        % \Ensure{flujo bloquenate, $g$}
        \State g=0
        \State flag=1
        \While {flag}
            \State $k = s$ (pivote)
            \State Inicializar el camino $p$ en $s$
            \While {$x \neq t$ and flag}
                \If {$\Gamma^{+}(x) \neq \emptyset$} AVANZAR$(x,p,g,NA)$
                \ElsIf {$x \neq s$} RETROCEDER$(x,p,g,NA)$
                \Else flag=0
                \EndIf
                \If {x == t} INCREMENTAR(g)
                \EndIf
            \EndWhile
        \EndWhile
        \State{\Return $g$}
    \end{algorithmic}
\end{algorithm}
% \begin{algorithm}
%     \caption{AVANZAR}
%     \begin{algorithmic}[1]
%         % ENTRADA / SALIDA
%         \Require{$x, p, g, NA$}
%         \State tomar $y \in \Gamma^{+}(x)$
%         \State agregar $y$ al camino $p$
%         \State $x = y$
%     \end{algorithmic}
% \end{algorithm}
% \begin{algorithm}
%     \caption{RETROCEDER}
%     \begin{algorithmic}[1]
%         % ENTRADA / SALIDA
%         \Require{$x, p, g, NA$} 
%         \State tomar $y =$ vertice anterior a $x$ en el camino
%         \State borrar el lado $\overrightarrow{yx}$ del NA
%         \State borrar $x$ del camino
%         \State $x = y$ 
%     \end{algorithmic}
% \end{algorithm}
% \begin{algorithm}
%     \caption{INCREMENTAR}
%     \begin{algorithmic}[1]
%         % ENTRADA / SALIDA
%         \Require{$G$} 
%         \State recorrer el camino $p$ para calcular $\epsilon$
%         \State aumentar $g$ en $\epsilon$ a lo largo de $p$
%         \State borrar lados saturados
%     \end{algorithmic}
% \end{algorithm}

Una vez que se tiene creado el camino para aumentar el flujo, el INCREMENTAR es 
$O(n)$ pues la longitud del camino es a la sumo $n$, e INCREMENTAR aumenta el 
flujo a lo largo de ese camino y borra todos los lados saturados. Además la 
complejidad de AVANZAR es $O(1)$ ya que consiste en buscar el primer vértice 
en la lista de $\Gamma^{+}(x)$, agregar un lado al camino y cambiar $x$. 
Por último, RETROCEDER también es $O(1)$ pues simplemente borramos un lado y 
cambiamos quien es $x$.

\begin{teorema} Complejidad de Dinitz-Even es $O(n^{2}m)$
\end{teorema}

\underline{Prueba}
\medskip

Como antes basta ver que la complejidad del \textbf{flujo bloqueante} es $O(nm)$. 
Una corrida es una "palabra" que se obtiene con DFS de la forma:
$$IA\ldots AAAIAARARR \ldots AA \ldots IA\ldots$$ 
donde:
\begin{itemize}
    \item $A \to AVANZAR \Rightarrow O(1)$
    \item $R \to RETROCEDER \Rightarrow O(1)$
    \item $I \to INCREMENTAR \Rightarrow O(r) = O(n)$
\end{itemize}

Calcular la complejidad de la palabla? Analizo un solo DFS hasta que no pueda 
AVANZAR, es decir, una subpalabra.
\medskip

$A \ldots AX$ con $X=I$ ó $X=R$, entonces la complejidad de cada palabra? Como 
hay $r$ niveles y cada $A$ incrementa el nivel del pivote, hay a lo sumo 
$r$ $A's$ seguidas, entonces:
$$\text{Complejidad}(A \ldots AX) = O(r) + complejidad(X)$$

$X$ puede ser $I$ o $R$, entonces si es $I$ es $O(r)$ y si es $R$ es $O(1)$. 
Entonces $$\text{Complejidad}(A \ldots AX)= \left\{ \begin{array}{lcc}
    O(r) + O(1) & \to & X=R \\
    \\O(r) + O(r)& \to & X=I\\
    \end{array}
    \right.$$

Entonces $=O(r) \Rightarrow O(n)$.
\medskip

Cuantas palabras hay? Si $X=R$, $R$ borra un lado y si $X=I$, $I$ borra al menos 
un lado. Entonces, cada $A \ldots Ax$ borra al menos un lado por lo que hay a los 
sumo $O(m)$ palabras ($\#$ lados del NA).
\medskip

Entonces son $O(m)$ palabras con complejidad $O(n)$ cada una, entonces $O(nm)$ 
total.$\square$

%%%%%% 4 %%%%%%
\begin{itemize}
    \item [4)] \textbf{Cual es la complejidad del algoritmo de Wave? Probarla. 
    (no hace falta probar que la distancia en networks auxiliares sucesivos 
    aumenta).}
    \label{dem:Wave}
\end{itemize}

La complejidad del algoritmo de Wave es $O(n^{3})$, pero como trabaja con NA's 
por el corolario \ref{cor:NA} basta ver que la complejidad de hallar un
\textbf{flujo bloqueante} es $O(n^{2})$.
\medskip

Para poder encontrar un flujo bloqueante, se hace una serie de ciclos de olas 
hacia adelante y olas hacia atrás. En cada ola hacia adelante hacemos una serie 
de forwardbalance (fwb) y en cada ola hacia atrás hacemos una serie de
backwardbalance (bwb). En cada ola hacia adelante (salvo en la última) al menos 
un vertice se bloquea, si los fwb no bloquean ningún vertice entonces todos 
quedan balanceados y es la última ola.
\medskip

Como los vertices nunca se desbloquean, hay a lo sumo $O(n)$ olas. Entonces tenemos 
que analizar dos casos, los fwb y los bwb y ver su complejidad. 
\medskip

En cada \textbf{fwb} vamos saturando lados salvo quizas uno. Entonces dividamos la 
complejidada en dos:
\begin{itemize}
    \item [1.] $S=$ complejidad total de los fwb saturado.
    \item [2.] $P=$ complejidad de fwb parcial.
\end{itemize}

$P$ es facil, en cada fwb solo se ejecuta una vez, entonces es $O(1)$ pero en una 
ola se pueden realizar $n-2$ fwb, entonces:
\begin{align*}
    P &= \#\,\text{total de fwb}\\
    &= \#\,\text{de fwb en una ola} * \,\text{olas}\\
    &= O(n) * O(n)\\
\end{align*}

En cuanto a $S$, cada vez que se ejecuta uno de esta parte, se borra un vecino de 
$\Gamma^{+}(x)$. Por lo tanto, será a lo sumo $O(d(x))$ en cada $x$, entonces:
$$s = \sum_{x} O(d(x)) = O(m)$$.
\medskip

Con \textbf{bwb} también lo dividimos en casos que vaciamos un lado $V$ y 
los que parcialmente vaciamos $Q$. El analisis de $Q$ es igual que el de $P$, es 
$O(1)$ en el bwb, por lo tanto
\begin{align*}
    Q &= \#\,\text{total de bwb(x)}\\
    &= \#\,\text{de bwx(x) en una ola} * \,\text{olas}\\
    &= O(n) * O(n)\\
\end{align*}

Y con respecto a $V$, cada vez que lo hacemos en un $bwb(x)$ borramos un 
vertice de $M(x)$ (vecinos hacia atrás). Por lo que esta acotado por:
$$O(\#\,\text{número de elementos máximos de $M(x)$}) = O(d(x)) = O(m)$$
Un detalle importante es que $\Gamma^{+}(x)$ es fijo y le voy sacando elementos y 
$M(x)$ al inicio es vacio, pero le voy agregando y luego removiendo vertices.
\medskip

Supongamos que $\overrightarrow{yx}$ se vacía, entonces $y$ se borra de $M(x)$,
pero más adelante podría ser que lo vuelva a poner en $M(x)$. Pero 
$\overrightarrow{yx}$ solo se puede vaciar si $x$ está bloqueado, pues de otra 
forma no pudiera devolverle flujo a $y$. Pero si $x$ esta bloqueado el 
vértice $y$ nunca más puede mandarle flujo. Entonces, si $\overrightarrow{yx}$ 
nunca más puede recibir flujo, menos aún va a poder volver a vaciarse. Entonces 
una vez que removó a $y$ no lo voy a poder agregar nuevamente a $M(x)$.
\medskip

Entonces, la complejidad final:
$$\text{Complejidad} = P + V + S + Q = O(n^{2}) + O(m) + O(n^{2}) + O(m) = O(n^{2}) \square$$

%%%%%% 5 %%%%%%
\begin{itemize}
    \item [5)] \textbf{Probar que la distancia en networks auxiliares sucesivos aumenta.}
    \label{dem:dist2}
\end{itemize}

\begin{teorema} (Dinitz) La cantidad de NA usados es $O(n)$
\end{teorema}
\underline{Prueba:}
\medskip

Sea NA un network auxiliar y NA' el siguiente network auxiliar. Sean $d_{i}(x) = d_{f}(s,x)$ 
las distancias de FF usadas para construir NA y $d'$ para NA'. Vamos a demostrar 
que $d(t) < d'(t)$.
\medskip

Sabemos por EK que $d \< d'$, acá queremos ver solo el $<$. Entonces, si 
$d'(t) = \infty$, ya está.
\medskip

Suponiendo, $d'(t) < \infty$, entonces existe al menos un camino aumentante (ca), 
entre $s$ y $t$ en el network original, por lo tanto existe un camino dirigido de 
$s$ a $t$ en NA'.
\medskip

Sea $s=x_{0},x_{1},\ldots,x_{n}=t$ un caminio dirigido entre $s$ y $t$ en NA', 
como NA' es un network por niveles, entonces $d(x_{i}) = i, i=0,\ldots,r$, pues 
mando flujo, ese camino no pude estar en NA, porque para pasar de NA a NA' se 
bloquean todos lo caminos de NA por lo tanto si ese camino estuviera en NA se 
hubiera bloqueado y no estaría en NA'. Sino es camino en NA entonces puede suceder:
\begin{itemize}
    \item [a)] Falta un vertice 
    \item [b)] Falta un lado
\end{itemize}

Veamos el a) entonces si falta un vertice, tomamos un $x$ cualquiera por 
lo que $x_{i} \notin NA \Rightarrow d(t) \< d(x_{i})$.
\medskip

Pero por EK, sabemos que $d \< d'$:
\begin{align*}
    d(t) \< d(x_{i}) \<_{EK} d'(x_{i}) &= i\\
    &<_{x\neq t} r = d'(t)\\
    &\Rightarrow d(t) < d'(t)
\end{align*}

Si falta al menos un lado (parte b) tengo dos posibles casos. Sea $i$ el primer 
indice tal que $\overrightarrow{x_{i}x_{i+1}} \notin NA$, con esto me aseguro 
que todos los anteriores esten.
\begin{itemize}
    \item [Caso 1.] $d(x_{i+1}) < i+1 \Rightarrow d(x_{i+1}) < d'(x_{i+1})$
        Entonces hacemos algo similar al caso a). pero con $x_{i+1}$.
        \begin{align*}
            d(t) &= d(x_{i+1}) + b(x_{i+1})\\
            &\< d(x_{i+1}) + b'(x_{i+1})\\
            &< i + 1 + b'(x_{i+1})\\
            &\text{por hip}\\
            &=d'(x_{i+1}) + b'(x_{i+1})\\
            &=d'(t) \Rightarrow d(t) < d'(t)\\
        \end{align*}
    \item [Caso 2.] 
        $d(x_{i+1}) \nless i+1$.
        Pero $d(x_{i+1}) \< i+1$, entonces $d(x_{i+1}) = i+1$. Como $i$ es el 
        primer indice con $\overrightarrow{x_{i}x_{i+1}} \notin NA$, entonces 
        $\overrightarrow{x_{j}x_{j+1}} \in NA\, \forall j<i$, como NA es por 
        niveles nos dice que la porción $x_{0}x_{1}\ldots x_{j}$ esta en NA.
        $\Rightarrow d(x_{j}) = j, \forall j<i$. Es decir, que vale 
        $\forall j \< i+1$.
        \medskip

        En particular $d(x_{i}) = i, d(x_{i+1}) = i+1$. Entonces, $x_{i}$ esta 
        en el nivel $i$ y $x_{i+1}$ en el nivel $i+1$. Entonces, 
        $\overrightarrow{x_{i+1}x_{i}} \notin NA$ por lo que vimos antes
        (los niveles no son legales).
        \medskip

        Como sabemos que $\overrightarrow{x_{i+1}x_{i}} \notin NA$ entonces 
        tenemos que no está ninguno de los dos, es decir, que 
        $\overrightarrow{x_{i}x_{i+1}} \notin NA$ y 
        $\overrightarrow{x_{i+1}x_{i}} \notin NA$.
        \medskip

        Pero $x_{i}$ pertenece al nivel $i$ y $x_{i+1}$ al nivel $i+1$, entonces
        $\overrightarrow{x_{i}x_{i+1}}$ podría estar, pero no puede mandar ni 
        devolver flujo. Como no está, caso fordward saturado o caso backward
        vacio. Pero $\overrightarrow{x_{i}x_{i+1}} \in NA'$, entonces se 
        tiene que haber des-saturado o llenado un poco según el caso al ir de 
        NA a NA'.
        \medskip

        Pero entonces, para pasar de NA a NA', tengo que haber usado el lado al 
        revés, pero no lo puedo haber usado porque $x_{i}x_{i+1}$ no existe.
\end{itemize}
$\square$

%%%%%% 6 %%%%%%
\begin{itemize}
    \item [6)] \textbf{Probar que el valor de todo flujo es menor o igual que la 
    capacidad de todo corte y que si f es un flujo, entonces las siguientes 
    afirmaciones son equivalentes:
    \begin{itemize}
        \item [i)] $f$ es maximal
        \item [ii)] Existe un corte $S$ tal que $v(f) = cap(S)$ (y en este caso, 
        $S$ es minimal)
        \item [iii)] No existen $f-$caminos aumentantes. (puede usar sin necesidad 
            de probarlo que si $f$ es flujo y $S$ es corte entonces 
            $v(f) = f(S,\overline{S})-f(\overline{S},S)$)
    \end{itemize}}
    \label{dem:valor}
\end{itemize}

\underline{Prueba} $3) \Rightarrow 2) \Rightarrow 1) \Rightarrow 3)$.
\medskip

$2) \Rightarrow 1)$ (si existe un corte $S$ tal que $v(f) = cap(S)$ 
entonces $f$ es maximal)
\medskip

Dados $S$ corte, $f$ y $g$ dos flujos, entonces sabemos que el valor de todo 
flujo es menor o igual a la capacidad de todo corte. Entonces $v(g) \< cap(S)$.
\medskip

Por hipotesis tenesmos que $v(f) = cap(S)$ entonces $v(g) \< cap(S) = v(f)$, 
por lo que $v(g) \< v(f)$ entonces $f$ es maximal. Además, si $T$ es un corte, 
$cap(T) \> v(f) = cap(S)$, entonces $S$ es minimal.

$1) \Rightarrow 3)$ (si $f$ es maximal entonces no existen $f-$caminos aumentantes)
\medskip

$\lnot\, 3) \Rightarrow \lnot\, 1)$ (contrareciproca)
\medskip

Entonces, existe un $f-$camino aumentante entre $s$ y $t$, entonces usando ese 
camino aumentante podemos construir un nuevo flujo con valor $f^{*}$ tal que 
$v(f^{*}) = v(f) + \epsilon$ esto implica que $f$ no es maximal, debido a que
$v(f^{*}) > v(f)$.
\medskip

$3) \Rightarrow 2)$, la idea es $\nexists \Rightarrow \exists$ (parte dificil)
\medskip

Definamos:
$$S = \llaves{s} \cup \llaves{x: \exists \, \text{un $f$ ca entre $s$ y $x$}}$$

Como $f$ es flujo y $S$ es corte, entonces:
$$v(f) = f(S,\overline{S}) - f(\overline{S},S)$$

Entonces, veamos primero
$$f(S,\overline{S}) = \sum_{x\in S, y\in \overline{S}, xy \in E} f(\overrightarrow{xy})$$

Y tomemos un par $(x,y)$ cualquiera con $x\in S, y \notin S, xy \in E$, entonces 
existe un $f-$camino aumentante entre $s\ldots x$ y como $y \notin S$ no existe 
$f-$ca entre $s\ldots y$.
\medskip

Supongamos que $f(\overrightarrow{xy}) < c(\overrightarrow{xy})$, no se satura, 
entonces el lado $\overrightarrow{xy}$ puede usarse en algún ca. Como $s\ldots x$ 
es $f-$ca, entonces $s\ldots xy$ es $f-$ca, entonces $y \in S$, lo que es un absurdo.
\medskip

$f(\overrightarrow{xy}) \nless c(\overrightarrow{xy}) \Rightarrow f(\overrightarrow{xy}) = c(\overrightarrow{xy})$ 
esto vale $\forall x\in S, y\notin S, \overrightarrow{xy}\in E$. Por lo tanto:
$$f(S,\overline{S}) = \sum_{x\in S, y\notin S, xy \in E} f(\overrightarrow{xy}) = \sum_{x\in S, y\notin S}c(\overrightarrow{xy}) = c(S,\overline{S}) = cap(\overline{S})$$

Por otro lado, si es un lado backward:
$$f(\overline{S},S) = \sum_{x\notin S, y\in S, xy \in E} f(\overrightarrow{xy})$$

tomemos un par $(x,y)$ cualquiera con $\overrightarrow{xy} \in E, x\notin S, y\in S$, 
entonces $\exists f-$ca, $s\ldots y$
\medskip

Supongamos que $f(\overrightarrow{xy}) > 0$, entonces podemos devolver flujo por 
$\overrightarrow{xy}$ por lo que:
$$\underbrace{s\ldots y}_{ca}\overleftarrow{x}$$ 

también es ca, entonces $x\in S$ lo que es un absurdo, entonces $f(\overrightarrow{xy}) = 0, \forall x\notin S, y\in S, \overrightarrow{xy}\in E$
\medskip

Entonces:
$$f(\overline{S},S) = \sum_{x\notin S, y\in S, xy\in E} f(\overrightarrow{xy}) = 0$$

Por lo tanto:
$$v(f) = f(S,\overline{S}) - f(\overline{S},S) = cap(S) - 0 = cap(S)$$
$$\square$$

%%%%%% 7 %%%%%%
\begin{itemize}
    \item [7)] \textbf{Probar que $2-$COLOR es polinomial.}
    \label{dem:2color}
\end{itemize}

% \textbf{Problema:} "k-color": Dado $G$ es $\chi(G) \< k$? $1-$color es trivial 
% (no es polinomial). $2-$color es polinoimal (hay algún algoritmo que corre en tiempo 
% polinomial en un rango de la entrada). Idea (la demostración por ahora no) Basta 
% correrlo para grafos conexos. Tomamos un $x\in V$, corremos BFD empezando en $x$. Si:
% \begin{align*}
%     N(z) &=\, \text{nivel $z$ en el arbol BFS}\\
%          &=\, \text{distancia entre $z$ y $x$ en el arbol BFS}\\
%          &=\, \text{distancia entre $z$ y $x$ en $G$}
% \end{align*}
% Sea: 
% \begin{align*}
%     C(z) &= (N(z)\, mod\, 2)\\
%     &IF(\text{$C$ es propio, return si, es $2-$colorable})\\
%     &ELSE(\text{return no es $2-$colorable})
% \end{align*}

% El algoritmo es polinomial porque BFS es $O(m)$ y chequear que es propio es $O(m)$,
% lo que hay que probar es si es correcto. Entonces supongamos que la respuesta es 
% "no es $2-$colorable". 

% $$\Rightarrow \exists v,z : c(v) = c(z) \wedge vz \in E$$

% Entonces $d(x,v) = d(x,z)\,\,mod\,\, 2$
% \medskip

% tomamos un camino entre $x$ y $v$ en BFS y un camino entre $x$ y $z$ en BFS y sea 
% $w$ el único vertice en común (como lo muestra la siguiente figura).

% \begin{figure}[H]
%     \centering
%     \tikzset{every picture/.style={line width=0.75pt}} %set default line width to 0.75pt        
%     \begin{tikzpicture}[x=0.75pt,y=0.75pt,yscale=-1,xscale=1]
%     %uncomment if require: \path (0,300); %set diagram left start at 0, and has height of 300
%     \centering
%     %Curve Lines [id:da9660361289294499] 
%     \draw    (185,174) .. controls (254,182) and (276,96) .. (315,54) ;
%     %Curve Lines [id:da45622019482482745] 
%     \draw    (285,98) .. controls (325,68) and (330,217) .. (370,187) ;
%     % Text Node
%     \draw (312,31.8) node [anchor=north west][inner sep=0.75pt]    {$x$};
%     % Text Node
%     \draw (380,180.8) node [anchor=north west][inner sep=0.75pt]    {$z$};
%     % Text Node
%     \draw (172,170.8) node [anchor=north west][inner sep=0.75pt]    {$v$};
%     % Text Node
%     \draw (270,86.8) node [anchor=north west][inner sep=0.75pt]    {$w$};
%     \end{tikzpicture}
% \end{figure}

% Miramos el ciclo en $G$: $w \ldots \underbrace{vz}_{\text{cruzo a $z$}} \ldots \underbrace{w}_{\text{vuelvo a $x$}}$ 
% (como lo muestra la siguiente imagen).
% \begin{figure}[H]
%     \centering
%     \tikzset{every picture/.style={line width=0.75pt}} %set default line width to 0.75pt        
%     \begin{tikzpicture}[x=0.75pt,y=0.75pt,yscale=-1,xscale=1]
%     %uncomment if require: \path (0,300); %set diagram left start at 0, and has height of 300
%     %Curve Lines [id:da9660361289294499] 
%     \draw    (185,174) .. controls (254,182) and (276,96) .. (315,54) ;
%     %Curve Lines [id:da45622019482482745] 
%     \draw    (285,98) .. controls (325,68) and (330,217) .. (370,187) ;
%     %Shape: Ellipse [id:dp8919263115782621] 
%     \draw   (164.05,194.88) .. controls (150.21,178.93) and (170.82,138.4) .. (210.07,104.35) .. controls (249.32,70.3) and (292.36,55.63) .. (306.19,71.58) .. controls (320.02,87.52) and (299.42,128.05) .. (260.17,162.1) .. controls (220.92,196.15) and (177.88,210.83) .. (164.05,194.88) -- cycle ;
%     %Shape: Ellipse [id:dp6731791355972647] 
%     \draw   (166.62,169.16) .. controls (168.21,153.5) and (224.06,146.33) .. (291.37,153.14) .. controls (358.68,159.96) and (411.96,178.18) .. (410.38,193.84) .. controls (408.79,209.5) and (352.94,216.67) .. (285.63,209.86) .. controls (218.32,203.04) and (165.04,184.82) .. (166.62,169.16) -- cycle ;
%     %Shape: Ellipse [id:dp4061401367808297] 
%     \draw   (265.02,75.48) .. controls (279.1,59.75) and (321.9,75.09) .. (360.61,109.74) .. controls (399.33,144.4) and (419.3,185.24) .. (405.22,200.97) .. controls (391.14,216.7) and (348.34,201.36) .. (309.63,166.71) .. controls (270.91,132.06) and (250.94,91.21) .. (265.02,75.48) -- cycle ;
%     % Text Node
%     \draw (312,31.8) node [anchor=north west][inner sep=0.75pt]    {$x$};
%     % Text Node
%     \draw (380,180.8) node [anchor=north west][inner sep=0.75pt]    {$z$};
%     % Text Node
%     \draw (175,164.8) node [anchor=north west][inner sep=0.75pt]    {$v$};
%     % Text Node
%     \draw (270,86.8) node [anchor=north west][inner sep=0.75pt]    {$w$};
%     \end{tikzpicture}
% \end{figure}

% Calculamos la longitud de este ciclo:
% \begin{align*}
%     longitud &= 1 + d(v,w) + d(z,w) \\
%     longitud\, mod\, 2 &= (1 + d(v,w) + \ldots + d(z,w)) \, mod\, 2\\
%     &= (1 + d(v,w) + d(z,w) + 2d(x,w))\, mod\, 2\\
%     &= (1 + d(x,v) + d(x,z))\, mod\, 2 = (1 + \overbrace{c(x) + c(z)}^{=0})\, mod\, 2\\
%     &= 1
% \end{align*}
% Es un ciclo impar, entonces no se puede colorear con 2 colores $\Rightarrow$ $\chi_(G) \> 3$.
% $\square$

Asumimos $G$ un grafo conexo. Sea $x \in V$ corramos $BFS(x)$, si al correr 
$BFS(x)$, cuando un vertice $z$ agrega a un $w$ agregrando al lado $zw$, tenemos 
un arbol. 
\medskip

Sea $N(z) = \text{Longitud del único camino entre $x$ y $z$ en el arbol BFS}$.
\medskip

Definimos:
$$C(z)=  N(z) \mod 2 =\left\{ \begin{array}{lcc}
    0   & \to & N(z)\,\, par \\
    \\1 & \to & N(z)\,\, impar\\
    \end{array}
    \right.$$

Si es propio $\Rightarrow \text{\textbf{return}}\,\, \chi(G) \< 2$.
\medskip

Si no propio $\Rightarrow \text{\textbf{return}}\,\, \chi(G) > 2$.
\medskip

\textbf{Fin del algoritmo.}
\medskip

Entonces, hay que demostrar $1)$ el algoritmo es polinomial, $2)$ el algoritmo
es correcto.
\medskip

Veamos $1)$ sabemos que $BFS(x)$ es $O(m)$ y chequear que el coloreo es propio 
es $O(m)$ entonces $O(m) + O(m) = O(m)$.
\medskip

$2)$ Supongamos que la respuesta es "no es $2-$colorable", es decir, que ningún 
otro coloreo con $2$ colores es propio lo que veremos es que $G$ tiene un ciclo 
impar.
\medskip

Entonces con $C$ no propio. $\exists z,w : C(z) = C(w) \wedge zw \in E$.
\medskip

Sea $x = z_{0}, z_{1}, \ldots , z_{r} = z$ un camino de $x$ a $z$ en el arbol BFS
y $x = w_{0}, w_{1}, \ldots , w_{t} = w$ otro camino de $x$ a $w$ en el arbol BFS.
\medskip

Además $\exists i : w_{i} = z_{i} \wedge w_{i+1} \neq z_{i+1}$
\medskip

Consideremos $w_{i}w_{i+1}, \ldots\, wz\, z_{r-1} \ldots z_{i}$, como se 
muestra en la siguiente figura.

\begin{figure}[H]
    \centering
        \tikzset{every picture/.style={line width=0.75pt}} %set default line width to 0.75pt        

        \begin{tikzpicture}[x=0.75pt,y=0.75pt,yscale=-1,xscale=1]
        %uncomment if require: \path (0,300); %set diagram left start at 0, and has height of 300

        %Curve Lines [id:da9660361289294499] 
        \draw    (185,174) .. controls (254,182) and (276,96) .. (315,54) ;
        %Curve Lines [id:da45622019482482745] 
        \draw    (285,98) .. controls (325,68) and (330,217) .. (370,187) ;
        %Shape: Ellipse [id:dp6731791355972647] 
        \draw   (169.98,154.25) .. controls (174.41,110.48) and (228.66,80.13) .. (291.15,86.45) .. controls (353.65,92.78) and (400.72,133.39) .. (396.29,177.16) .. controls (391.86,220.93) and (337.6,251.28) .. (275.11,244.96) .. controls (212.62,238.63) and (165.55,198.02) .. (169.98,154.25) -- cycle ;
        %Shape: Brace [id:dp29047934988211077] 
        \draw   (405,191) .. controls (409.33,189.25) and (410.61,186.21) .. (408.86,181.88) -- (386.24,126.14) .. controls (383.73,119.96) and (384.64,115.99) .. (388.97,114.24) .. controls (384.64,115.99) and (381.23,113.78) .. (378.72,107.6)(379.85,110.38) -- (356.11,51.86) .. controls (354.36,47.53) and (351.32,46.25) .. (346.99,48) ;
        %Shape: Brace [id:dp17333224647215562] 
        \draw   (254,36) .. controls (250.23,33.25) and (246.97,33.75) .. (244.22,37.52) -- (208.24,86.79) .. controls (204.31,92.18) and (200.46,93.49) .. (196.69,90.74) .. controls (200.46,93.49) and (200.37,97.56) .. (196.44,102.94)(198.21,100.52) -- (160.47,152.21) .. controls (157.72,155.98) and (158.22,159.24) .. (161.99,161.99) ;

        % Text Node
        \draw (312,31.8) node [anchor=north west][inner sep=0.75pt]    {$x$};
        % Text Node
        \draw (380,180.8) node [anchor=north west][inner sep=0.75pt]    {$z$};
        % Text Node
        \draw (175,164.8) node [anchor=north west][inner sep=0.75pt]    {$w$};
        % Text Node
        \draw (233,90.8) node [anchor=north west][inner sep=0.75pt]    {$w_{i} =z_{i}$};
        % Text Node
        \draw (174,78.8) node [anchor=north west][inner sep=0.75pt]    {$t$};
        % Text Node
        \draw (401,103.8) node [anchor=north west][inner sep=0.75pt]    {$r$};
        % Text Node
        \draw (214,136.8) node [anchor=north west][inner sep=0.75pt]    {$t-i$};
        % Text Node
        \draw (334,142.8) node [anchor=north west][inner sep=0.75pt]    {$r-i$};
    \end{tikzpicture}
\end{figure}

Como $w_{i} = z_{i}$ es un ciclo con

$$t - i + r - i + \underbrace{1}_{zx} = t + r - 2i + 1 \,\, \text{lados}$$

Pero $2i$ es par y $t+r$ también es par pues $C(z) = r \mod 2$ y $C(w) = t \mod 2$ 
y como $C(z) = C(w)$ entonces $r \mod 2 = t \mod 2$. Entonces la suma de los 
lados es impar. $\square$

%%%%%% 8 %%%%%%
\begin{itemize}
    \item [8)] \textbf{Enunciar y probar el Teorema de Hall.}
    \label{dem:hall}
\end{itemize}

\begin{teorema} (Hall) Si $G = (\bar{X} \cup \bar{Y}, E)$ es bipartito con partes 
    $\bar{X}$ e $\bar{Y}$, entonces existe matching completo de $\bar{X}$ en $\bar{Y}$
    si y solo si $|S| \< |\Gamma(S)| \,\, \forall S \subseteq \bar{X}$.
\end{teorema}

\underline{Prueba:}
\medskip

$(\Rightarrow)$ Si existe un matching completo de $\bar{X}$ en $\bar{Y}$, el matching 
induce una función inyectiva de $\bar{X} \cap \bar{Y}$ tal que $\psi(x) \in E$, 
por lo tanto:
$$\psi(S) \subseteq \Gamma(S)$$

por lo tanto, $|\Gamma(S)| \> |\psi(S)| = |S|$
\medskip

$(\Leftarrow)$ Queremos ver que si $\underbrace{|S| \< |\Gamma(S)|}_{P} \Rightarrow \underbrace{\exists\,\, \text{un matching completo de $\bar{X} \cap \bar{Y}$}}_{Q}$.
\medskip

Lo veremos de la siguiente forma $[P \Rightarrow Q] = [\neg Q \Rightarrow \neg P]$ 
por contrareciproca.
\medskip

\textbf{Si $\nexists$ matching completo de $\bar{X}$ en $\bar{Y}$, entonces al 
correr el algoritmo llegamos a un matching maximal que no cubre a $\bar{X}$}. Que 
es equivalente a hallar un flujo maximal entero $f$ cuyo valor no es $|\bar{X}|$.
\medskip

Al hallar $f$, también hallamos un corte minimal que vamos a denotar por $c$ (seria 
la última cola, al correr E-K).
\medskip

Sea $S = c \cap \bar{X}$, $T = c \cap \bar{Y}$, como en la siguiente imagen $(
T \neq t)$.
\begin{figure}[H]
    \centering
    \tikzset{every picture/.style={line width=0.75pt}} %set default line width to 0.75pt        

    \begin{tikzpicture}[x=0.75pt,y=0.75pt,yscale=-1,xscale=1]
    %uncomment if require: \path (0,345); %set diagram left start at 0, and has height of 345
    
    %Shape: Circle [id:dp14718267969579513] 
    \draw   (118,156) .. controls (118,121.21) and (146.21,93) .. (181,93) .. controls (215.79,93) and (244,121.21) .. (244,156) .. controls (244,190.79) and (215.79,219) .. (181,219) .. controls (146.21,219) and (118,190.79) .. (118,156) -- cycle ;
    %Shape: Circle [id:dp9130235045197983] 
    \draw   (133,164.5) .. controls (133,142.68) and (150.68,125) .. (172.5,125) .. controls (194.32,125) and (212,142.68) .. (212,164.5) .. controls (212,186.32) and (194.32,204) .. (172.5,204) .. controls (150.68,204) and (133,186.32) .. (133,164.5) -- cycle ;
    %Shape: Circle [id:dp5449515002070981] 
    \draw   (272,159) .. controls (272,124.21) and (300.21,96) .. (335,96) .. controls (369.79,96) and (398,124.21) .. (398,159) .. controls (398,193.79) and (369.79,222) .. (335,222) .. controls (300.21,222) and (272,193.79) .. (272,159) -- cycle ;
    %Shape: Circle [id:dp4453429157761797] 
    \draw   (287,167.5) .. controls (287,145.68) and (304.68,128) .. (326.5,128) .. controls (348.32,128) and (366,145.68) .. (366,167.5) .. controls (366,189.32) and (348.32,207) .. (326.5,207) .. controls (304.68,207) and (287,189.32) .. (287,167.5) -- cycle ;
    
    % Text Node
    \draw (138,76.8) node [anchor=north west][inner sep=0.75pt]    {$\overline{X}$};
    % Text Node
    \draw (337,75.8) node [anchor=north west][inner sep=0.75pt]    {$\overline{Y}$};
    % Text Node
    \draw (168,159.8) node [anchor=north west][inner sep=0.75pt]    {$S$};
    % Text Node
    \draw (315,157.8) node [anchor=north west][inner sep=0.75pt]    {$T$};
    
    
    \end{tikzpicture}
\end{figure}

Entonces, $T$ forma parte de $c$ 
por lo tanto forma parte de la última cola. Entonces todos sus elementos fueron 
agregados por alguien (pues $S \notin T$), ese alguien debe ser vecino, y como el 
grafo es bipartito y $T \subseteq \bar{Y}$, esos vecinos deben estar en $\bar{X}$.
\medskip

Pero además deben haber estado en la cola, es decir, que están en $c$. Entonces 
el vecino estaba en $S$. En resumen:
\medskip

$\forall y\in T, \exists x\in S: xy \in E \Rightarrow$ si $xy$ es lado $\Rightarrow$
$y$ es vecino $x$ con $x\in S$. Entonces, $y\in \Gamma(x) \subseteq \Gamma(S) \Rightarrow$
\begin{equation}
    T \subseteq \Gamma(S)
    \label{eq:1hall}
\end{equation}
Pero además:
\begin{equation}
    \Gamma(S) \subseteq T
    \label{eq:2hall}
\end{equation}

Ahora probemos \ref{eq:2hall}, para ver si se cumple la igualdad. Sea 
$y\in \Gamma(S) \Rightarrow \exists x\in S: xy\in E$ ($x$ esta en la cola).
\medskip

Supongamos que si $f(\overrightarrow{xy}) = 0$ entonces $x$ puede agregar a $y$ 
a la cola entonces $y \in T$.
\medskip

Supongamos que si $f(\overrightarrow{xy}) = 1$ entonces $x$ no puede agregar a $y$ 
a la cola pero $x\in S$ entonces algún vertice $z$ agrego a $x$ a la cola.
\medskip

Como $f(\overrightarrow{xy}) = 1$ entonces $out_{f}(x)=1 \Rightarrow In_{f}(x)=1 \Rightarrow f(\overrightarrow{sx})=1$.
Entonces, $s$ no agrego a $x$ a la cola por lo que $z \neq s$.
\medskip

Como $z$ debe ser un vecino de $x$ y $z \neq s$ entonces $z \in \bar{Y}$. Y para 
agregar a $x$, lo debe haber hecho backward, entonces $f(\overrightarrow{xz})=1$ 
sino no puede. Entonces: $f(\overrightarrow{xy})=1$ y $f(\overrightarrow{xz}) = 1$, 
por lo que $y=z$.
\medskip

Pero si $y=z$, agrego a $x$ a la cola, y esta en $c$, entonces 
$y\in c \cap \bar{Y} = T$, con esto \ref{eq:2hall} queda demostrado.
\medskip

Además, \ref{eq:1hall} y \ref{eq:2hall} $\Rightarrow$ 
\begin{equation}
    T = \Gamma(S)
    \label{eq:3hall}
\end{equation}

Sea $S_{0} = \llaves{x\in \bar{X}: In_{f}(x)=0}$, entonces $s$ agrega a los vertices 
de $S_{0}$ a la cola, entonces:
\begin{equation}
    S_{0} \subseteq S
    \label{eq:4hall}
\end{equation}

Como estamos suponiendo que $v(f) \neq |\bar{X}|$, entonces:
\begin{equation}
    S_{0} \neq \emptyset
    \label{eq:5hall}
\end{equation}
(me queda al menos un vertice sin matchear).
\medskip

Queremos comparar $S - S_{0}$ con $T$.
\medskip

$y\in T \Leftrightarrow y$ es puesto en la cola por alguien pero como $t\notin c$, 
y no puede poner a $t$. Entonces, $f(\overrightarrow{yt})=1 \Rightarrow out_{f}(y)=1$ 
y $In_{f}(y)=1$. Entonces $\exists x: f(\overrightarrow{xy})=1$.
\medskip

Como $f(\overrightarrow{xy})=1$ entonces, $y$ agrega a $x$ a la cola y $x\in S$ 
además $out_{f}(x) =1 \Rightarrow In_{f}(x)=1 \Rightarrow x \notin S_{0}$. 
Entonces, podemos concluir que $x\in S-S_{0}$. Y hay un solo $x$ con 
$f(\overrightarrow{xy})=1$.
\medskip

Entonces, tengo una función $y\to x$ y $T$ a $S-S_{0}$ inyectiva, pero además 
si $x\in S-S_{0}$ entonces $in_{f}(x)=1 \Rightarrow out_{f}(x)=1 \Rightarrow \exists y: f(\overrightarrow{xy})=1$.
\medskip

Entonces $y\in \Gamma(S) = T$ entonces tengo una biyección entre 
\begin{equation}
    T\,\, \text{y}\,\, S-S_{0}
    \label{eq:6hall}
\end{equation}
Finalmente:

\begin{equation}
    |\Gamma(S)| \underbrace{=}_{\ref{eq:3hall}} |T| \underbrace{=}_{\ref{eq:6hall}} |S-S_{0}| \underbrace{=}_{\ref{eq:4hall}} |S| - |S_{0}| < |S|
\end{equation}

La última desigualdad vale pues $S_{0} \neq \emptyset$. Hemos probado que 
$|\Gamma(S)| < |S|$, lo que contradice la hipótesis de que $|S| \< |\Gamma(S)|$.
para todo $S \subseteq \bar{X}$.
%%%%%% 9 %%%%%%
\begin{itemize}
    \item [9)] \textbf{Enunciar y probar el teorema del matrimonio de Konig}
    \label{dem:konig}
\end{itemize}

\begin{teorema} (matrimonio de konig) Todo grafo bipartito regular tiene un matching 
    perfecto (todos los vertices forman parte del matching).
\end{teorema}

\underline{Prueba:} Dado un conjunto de vertices definimos:

$$E_{w} = \llaves{zw\in E: w\in W}$$

Sean $\bar{X}$ e $\bar{Y}$ las partes de $G$ y supongamos $W \subseteq \bar{X}$ 
(completamente contenido en $\bar{X}$). Entonces:
\begin{align*}
    |E_{w}| &= |\llaves{zw: w\in W}| \\
    &\text{Como $W\subseteq \bar{X}$ y no hay lados entre vertices de $\bar{X}$},\\ 
    &\text{cada lado que aparece en $E_{w}$, aparece una sola vez} \\
    &=\sum_{w\in W} |\Gamma(w)|\\
    &=\sum_{w\in W} \delta(w)\\
    &\text{Como $G$ es regular}\\
    &\sum_{w\in W} \Delta = \Delta |W|
\end{align*}
Si $W \subseteq \bar{Y}$ vale el mismo analisis y tambien tenemos que $|G_{w}| = \Delta |W|$.
\medskip

Primero demostraremos que hay un matching completo (basta demostrar la condición 
de Hall).
\medskip

Sea $S \subseteq \bar{X}$ y sea $l \in E_{s}$, entonces $l$ es de la forma $l=xy$ 
con $x\in S$ y $y\in \bar{Y}$. Como $l$ es lado, entonces $y$ es vecino de $x$ 
entonces $y\in \Gamma(x)$. Pero $x\in S \Rightarrow y\in \Gamma(S)$. Como $l=xy$ 
entonces si $y\in \Gamma(S) \Rightarrow l\in E_{\Gamma(S)}$.
\medskip

Conclusión: $E_{S} \subseteq E_{\Gamma(S)} \Rightarrow |E_{s}| \< |E_{\Gamma(S)}|$. 
Y como $S \subseteq \bar{X} \Rightarrow |E_{S}| = \Delta |S|$. A su vez, 
$\Gamma(S) \subseteq \bar{Y} \Rightarrow |E_{\Gamma(S)}| = \Delta |\Gamma(S)|$. 
Entonces podemos decir que 
$\cancel{\Delta} |S| \< \cancel{\Delta} |\Gamma(S)| \Rightarrow |S| \< |\Gamma(S)|$.
\medskip

Entonces se satisface la condición de Hall, podemos afirmar que existe matching 
completo de $x$ a $y$. Para ver que ese matching es perfecto basta ver que 
$|\bar{X}| = |\bar{Y}|$.
\medskip

Como $G$ es bipartito los unicos lados son entre $x$ e $y$. Entonces:
\begin{align*}
    E &= E_{\bar{X}} + E_{\bar{Y}}\\
    &\text{Por lo tanto}\\
    &|E_{\bar{X}}| = |E_{\bar{Y}}|\\
    &\text{Como $G$ es regular}\\
    &\Delta |\bar{X}| = \Delta |\bar{Y}|\\
    &|\bar{X}| = |\bar{Y}|
\end{align*}

%%%%%% 10 %%%%%%
\begin{itemize}
    \item [10)] \textbf{Probar que si $G$ es bipartito entonces 
    $\chi^{'}(G) = \Delta(G)$}
    \label{dem:chi}
\end{itemize}

\begin{corolario} (tambien de koring) $G$ bipartito entonces $\chi^{'}(G) = \Delta(G)$.
\end{corolario}

Donde $\chi^{'}(G)$ es el indice cromatico (lados del un grafo), osea, la menor 
cantidad de colores necesarios para colorear los lados de un grafo de forma tal 
que lados con un vertice en común tengan colores distintos. Es obvio que $\Delta \< \chi^{'}(G)$.
\begin{figure}[H]
    \centering
    \tikzset{every picture/.style={line width=0.75pt}} %set default line width to 0.75pt        
    \begin{tikzpicture}[x=0.75pt,y=0.75pt,yscale=-1,xscale=1]
    %uncomment if require: \path (0,300); %set diagram left start at 0, and has height of 300
    
    %Straight Lines [id:da7690582805815367] 
    \draw    (201,99) -- (200,219) ;
    %Straight Lines [id:da936663735130838] 
    \draw    (220,112) -- (219,220) ;
    %Straight Lines [id:da7530223958992941] 
    \draw    (241,107) -- (240,220) ;
    %Straight Lines [id:da020592477605207327] 
    \draw    (258,107) -- (257,220) ;
    \end{tikzpicture}
\end{figure}

\underline{Prueba:} Supongamos $G$ regular, por el teorema del matrimonio $G$ 
tiene un matching perfecto, puedo colorear a todos con un color sin ningún 
problema, ej. puedo usar el color $1$.
\medskip

Luego remuevo los lados, entonces $\widetilde{G} = G -\text{esos lados}$, cada 
vertice disminuye su grado en $1$. Entonces $\widetilde{G}$ sigue siendo regular. 
Por lo que tiene un matching perfecto. Coloreo esos lados con el color $2$, los 
remuevo, etc. Y así sucesivamente.
\medskip

Siempre va a ser regular hasta el final y terminamos coloreando con $\Delta$ 
colores. 

\begin{lema} $G$ bipartito, entonces existe $H$ bipartito regular tal que 
    $G \subseteq H$. Entonces $\Delta(G) = \Delta(H)$.
\end{lema}

\underline{Prueba:} Si $G$ es un grafo bipartito, entonces podemos encontrar un 
grafo bipartito regular $H$ tal que $G \subseteq H$. Esto se debe a que un grafo 
bipatito regular es un grafo en el que todos los vertices tienen el mismo número 
de vecinos. Para encontrar un grafo que contenga a $G$, podemos comenzar por 
elegir cualquier conjunto de vértices de $G$ que formen un conjunto 
completo. Luego, podemos agregar vértices adicionales al grafo uno por uno, 
asegurandonos de que todos los nuevos vértices estén conectados a todos los 
vértices existentes. El grafo resultante será un grafo bipartito tegular que 
contiene a $G$.
\medskip

Una vez que hemos encontrado un grafo bipartito regular $H$ que contenga a $G$, 
podemos mostrar que $\Delta(G) = \Delta(H)$. Esto se debe a que todos los 
vértices de $G$ tienen el mismo número de vecinos en $H$. Por lo tanto, el 
grado de cualquier vertice en $G$ debe ser igual al grado de cualquier 
vertice de $H$. Esto significa que $\Delta(G) = \Delta(H)$.
\medskip

Entonces $\chi^{'}(H) = \Delta$. $G \subseteq H \Rightarrow \chi^{'}(G) \< \Delta$. 
Como $\Delta \< \chi^{'}(G) \Rightarrow \chi^{'}(G) = \Delta$.

%%%%%% 11 %%%%%%
\begin{itemize}
    \item [11)] \textbf{Probar la complejidad $O(n^{4})$ del algoritmo Hungaro y 
    dar una idea de como se la puede reducir a $O(n^{3})$.}
    \label{dem:hungaro}
\end{itemize}

\begin{teorema} La complejidad del algoritmo Hungaro como lo vimos en clase es $O(n^4)$.
\end{teorema}

\underline{Prueba:} 
Se corre EK sobre la matriz y, crucialmente, no tenemos que cambiar ningún network 
al cambiar la matriz, y vimos que al cambiar la matriz se pueden perder algunos 
ceros pero ninguno del matching parcial que tenemos hasta el momento.
\medskip

Como no se pierde ningún cero del matching parcial que tenemos, podemos continuar 
con la busqueda como veniamos, simplemente revisando las columnas donde hay nuevos 
ceros si lo hacemos completamente con el sistema de matriz o bien agregando esas 
columnas a la cola si lo hacemos con colas explícitas.
\medskip

De esta forma continuaremos con varios posibles cambios de matriz en el medio, 
pero terminaremos agregando UN lado extra al matching.
\medskip

Restar el $min$ de una fila es $O(n)$, entonces restar $min$ de 
cada fila es $O(n^2)$. Es el mismo análisis para las columnas 
es $O(n^{2})$.
\medskip

Hallar matching inicial es $O(n^2)$ pues hay que revisar $n$ filas y para cada 
una de las $n$ columnas. Este matching se puede extender a lo sumo 
$O(n)$ veces. Por lo tanto:
$$\text{Complejidad Hungaro} = O(n^2) + O(n)*(\text{Complejidad de extender matching en un lado})$$

Para extender el matching, hay que revisar $n$ filas buscando ceros, entonces cada 
busqueda es $O(n)$, por lo tanto en el peor de los casos si revisamos todas
las filas hasta poder extender el matching es $O(n^{2})$.
\medskip

Luego de revisar a lo sumo $n$ filas, si o si extendemos el matching pero 
en el medio quizas debamos hacer cambios de matrices, si seguimos con el matching 
parcial que teniamos, no hace falta re-escanear las filas de $S$ (esta dentro del 
cambio de matriz).
\medskip

Y además por el lema sabemos que hay $\< O(n)$ cambios de matries antes de extender 
el matching. Por lo tanto:
$$\text{Complejidad de extender el matching en un lado} = \underbrace{O(n^{2})}_{\text{escanear filas}} + \underbrace{O(n)}_{\text{\# cambio de matrices}} * CCM$$

donde $CCM$ es la complejidad de hacer un cambio de matriz (esto pude cambiar al programarlo).
Entonces, cambiar la matriz requiere:
\medskip

\begin{itemize}
    \item [1.] Calcular $m = \min \llaves{S \times \overline{\Gamma(S)}}$ (esto es $O(|S \times \overline{\Gamma(S)}|)=O(n^2)$).
    \item [2.] Restar $m$ de $S$ (filas) esto es $O(n)*|S| = O(n^2)$. Sumar $m$ a $\Gamma(S)$ (columnas) 
        eso es $O(n)*|\Gamma(S)| = O(n^2)$.
\end{itemize}
Entonces:
$$CCM = O(n^2) + O(n^2) + O(n^2) = O(n^2)$$

Por lo tanto:
$$\text{Complejidad hungaro} = O(n^{2}) + O(n)*(O(n^{2}) + O(n)*O(n^{2})) = O(n^{4})$$

\begin{teorema} El hungaro se puede coficar en $O(n^3)$.
\end{teorema}
\underline{Prueba:}
\medskip

Hay que tratar de hacer que $CCM = O(n)$, entonces:
$$\text{Complejidad hungaro} = O(n^{2}) + O(n)*(O(n^{2}) + O(n)*O(n)) = O(n^{3})$$

Debemos:
\begin{itemize}
    \item [1.] Hallar un $min$ de $O(n^{2})$ elementos en $O(n)$.
    \item [2.] Debemos cambiar $O(n^{2})$ elementos en $O(n)$.
\end{itemize}

Para hacer $1.$ parte del costo de hallar cada $m$ se traslada a la parte donde 
escaneamos filas, entonces:
\begin{align*}
    m &= \min \llaves{S \times \overline{\Gamma(S)}},\,\, \text{Sea $C$ la matriz de costos} \\
    &= \min \llaves{C_{x,y} : x \in S, y \notin \Gamma(S)} \\
    &= \min_{y\notin \Gamma(S)} \parent{\min_{x\in S}\llaves{C_{x,y}}}\,\,(\text{esto de puede calcular en $O(n)$ haciendo})\\
    &= \min_{y \notin \Gamma(S)} M_{y} 
\end{align*}

donde $M_{y} = \min_{x\in S}\llaves{C_{x,y}}$, siempre hay que tenerlo precalculado. Precalcular 
los $M_{y}$ demanda $O(n)$ pero los podemos hacer cuando escanemos una fila al 
buscar cero, en los no cero actualizamos $M_{y}$.
\medskip

Para $2.$ hacemos una resta y suma virtual (indicar cuanto se le deberia restar y sumar). 
Entonces usamos $RF(x)$ que indique cuanto restarle a la fila $x$ y $SC(x)$ que indique 
cuanto sumarle a la columna $y$.
\medskip

Restar $m$ de $S$ es simplemente hacer $RF(x) -= m,\, \forall x\in S$ esto es $O(n)$.

Sumar $m$ a $\Gamma(S)$ es simplemente hacer $SC(y) += m,\, \forall y\in \Gamma(S)$ esto es $O(n)$.
\medskip

El problema es el chequeo de ceros. Entonces, en vez de hacer
\begin{lstlisting}[language=C]
    if(0 == C(x)(y))
\end{lstlisting}

se chequea haciendo:
\begin{lstlisting}[language=C]
    if(0 == C(x)(y) - RF(x) + SC(y))
\end{lstlisting}
que es $O(1)$.

%%%%%% 12 %%%%%%
\begin{itemize}
    \item [12)] \textbf{Enunciar el teorema de la cota de Hamming y probarlo}
    \label{dem:hamming}
\end{itemize}
\begin{teorema} (cota de hamming) Sea $C$ código de longitud $n$, $t = \lfloor \frac{\delta -1}{2} \rfloor$, entonces:
    $$|C| \< \frac{2^{n}}{1+n+ \binom{n}{2} + \cdots + \binom{n}{t}}$$
\end{teorema}
\underline{Prueba:}

Sea $A = \bigcup_{x\in C} D_{t}(x)$. Como $C$ corrige $t$ errores, 
$D_{t}(x) \cap D_{t}(y) = \emptyset,\, \forall x,y\in C, x\neq y$. Entonces esa 
unión es disjunta por lo que $|A| = \sum_{x\in C} |D_{t}(x)|$.
\medskip

Queremos calcular $|D_{t}(x)|$, para ello definimos $S_{r}(x) = \llaves{y\in \llaves{0,1}^{n}: d_{H}(x,y)=r}$ 
(parto al disco en circulos de radio $r$).
\medskip

Entonces, $D_{t}(x) = \bigcup_{r=0}^{t} S_{r}(x)$ y la unión es disjunta. Entonces 
$$|D_{t}(x)| = \sum_{r=0}^{t} |S_{r}(x)|$$

$y\in S_{r}(x) \iff y\,\, \text{difiere de $x$ en exactamente $r$ lugares}$.
\medskip

$y\in S_{r}(x) \to r$ lugares ($r$ posiciones del conjunto $\llaves{1,2, \ldots, n}$), 
es un biyección.
\medskip

Cada $y\in S_{r}(x)$ determina $r$ lugares y el conjunto de $r$ lugares determina 
un $y\in \delta(x)$. Entonces:
$$|S_{r}(x)| = |\llaves{L \subseteq \llaves{1,\ldots,n}:|L|=r}| = \binom{n}{r}$$

(es cantidad de conjuntos de subconjuntos). Por lo tanto:
\begin{align*}
    |A| &= \sum_{x\in C} |D_{t}(x)| = \sum_{x\in C} \sum_{r=0}^{t} |S_{r}(x)|\\
    &= \sum_{x\in C} \sum_{r=0}^{t} \binom{n}{r}\\
    &= \parent{\sum_{r=0}^{t} \binom{n}{r}} \cdot |C|\\
\end{align*}
Despejando $C$:
$$|C| = \frac{|A|}{\sum_{r=0}^{t} \binom{n}{r}}\< \frac{2^{n}}{\sum_{r=0}^{t} \binom{n}{r}}$$

la desiguandad sale de que $A = \bigcup_{x\in C} D_{t}(x) \subset \llaves{0,1}^{n}$.
$\square$

%%%%%% 13 %%%%%%
\begin{itemize}
    \item [13)] \textbf{Probar que si $H$ es matriz de chequeo de $C$, entonces
        $$\delta(C) = \min \llaves{j: \exists \,\,\text{un conjunto de $j$ columnas LD de $H$}}$$
        (LD es linealmente dependiente)}
    \label{dem:delta}
\end{itemize}

Un código es lineal si es un subespacio vectorial de $\llaves{0,1}^{n}$.

\begin{definition} El peso de Hamming es $|x|=d_{H}(x,0)=\# \, \text{de 1's de $x$}$.
\end{definition}

\underline{Propiedad} $C$ lineal $\Rightarrow \delta(C) = \min \llaves{|v|: v\in C, v\neq 0}$.
\medskip

\underline{Prueba:}
\begin{teorema} Si $H$ es matriz de chequeo de $C$, entonces:
    \begin{align*}
        \delta(C) &= \min\,\,\llaves{\text{número de columnas de $H$ que son LD}}\\
        &= \min \llaves{j: \exists j\,\text{columnas LD de $H$}}
    \end{align*}
\end{teorema}

\underline{Prueba:}
\medskip

Sea $m = \min \llaves{j: \exists j\,\text{columnas LD de $H$}}$. Probaremos 
$\delta(C) \< m$ y luego $\delta(C) \> m$. Denotaremos la $j-$ésima columna de $H$ 
por $H^{(j)}$. Por definición de $m$ existe $j_{1},\ldots j_{m}$ tal que 
$H^{(j_{1})},\ldots, H^{(j_{m})}$ son LD.
\medskip

Entonces existen $c_{1},\ldots, c_{m}$ no todos $0$ tales que:
$$c_{1}H^{(j_{1})} + \cdots + c_{m}H^{(j_{m})} = 0$$

Sea $e_{i} = (0,\ldots,0,1,0,\ldots,0)$ con $1$ en la posición $i$. Entonces:
\begin{equation*}
    He_{i}^{t} =
    \begin{bmatrix}
        &   & i      &   &   & \\
        &   & \vdots &   &   & \\
        &   & i      &   &   & \\
        &   & \vdots &   &   & \\
        &   & i      &   &   & \\
    \end{bmatrix}
    \begin{bmatrix}
        0\\
        \vdots\\
        1\\
        \vdots\\
        0
    \end{bmatrix}
    =
    H^{(i)}
\end{equation*}

Es la columna $i-$ésima de $H$.
\medskip

Sea $x = c_{1}e_{j_{1}}+c_{2}e_{j_{2}}+\ldots + c_{m}e_{j_{m}}$, como no todos
los $c_{j}$ son $0$, entonces $x \neq 0$, por lo que:
\begin{align*}
    Hx^{t} &= H(c_{1}e_{j_{1}}^{t}+c_{2}e_{j_{2}}^{t}+\ldots + c_{m}e_{j_{m}}^{t})\\
    &= c_{1}He_{j_{1}}^{t}+c_{2}He_{j_{2}}^{t}+\ldots + c_{m}He_{j_{m}}^{t} \\
    &= c_{1}H^{(j_{1})} + c_{2}H^{(j_{2})} +\ldots + c_{m}H^{(j_{m})} = 0\\
    &\Rightarrow Hx^{t} = 0 
\end{align*}

$\Rightarrow x\in C$ pues $C =Nu(H)$, pero su peso es $\< m$, pues es suma de a 
lo sumo $m$ vectores de peso $1$. y vimos que $x \neq 0$ entonces 
sabemos que:
$$\delta(C) = \min \llaves{|v| : v\in C, v\neq 0}$$
Por lo tanto como $x \in C$ y $x \neq 0$, entonces $x$ está en ese conjunto, así 
que tenemos:
$$\delta(C) = \min \llaves{|v| : v\in C, v\neq 0} \< |x|$$
Pero su peso es $\< m$, entonces:
$$\delta(C) = \min \llaves{|v| : v\in C, v\neq 0} \< |x| \< m$$
Entonces, $\delta(C) \< m$.
\medskip

La otra parte de la desigualdad. Sea $x\neq 0 : \delta(C) = |x|$, entonces:
\begin{align*}
    x &= c_{1}e_{i_{1}} + \ldots + c_{\delta(C)}e_{i_{\delta(C)}}\\
    =&\llaves{\text{como $x\in C$, entonces $x\neq 0$}}\\
    0 &= Hx^{t} = H^{(i_{1})} + \ldots + H^{(i_{\delta(C)})}\\
    &\Rightarrow \llaves{H^{j},\ldots, H^{(i_{\delta(C)})}}\,\text{son LD} \Rightarrow m \< \delta(C)
\end{align*} $\square$

%%%%%% 14 %%%%%%
\begin{itemize}
    \item [14)] \textbf{Sea $C$ un código cíclico de dimensión $k$ y longitud $n$ y sea $g(x)$ 
    su polinomio generador. Probar que:
    \begin{itemize}
        \item [i)] $C$ esta formado por los multiplos de $g(x)$ de grado menor 
            que $n$.
            $$C = \llaves{p(x):gr(p) < n \,\,\&\,\, g(x) | p(x)}$$
        \item [ii)] $C = \llaves{v(x) \odot g(x) : \,\, \text{$v$ es un polinomio cualquiera}}$
        \item [iii)] $gr(g(x)) = n - k$
        \item [iv)] $g(x)$ divide a $1+x^{n}$
    \end{itemize}}
    \label{dem:polinomio}
\end{itemize}

\begin{teorema} (Fundamental de código ciclico) Sea $C$ un código ciclico de 
    longitud $n$ con generador $g(x)$ entonces:
    \begin{itemize}
        \item [1)] $C = \llaves{p(x)\in \mathbb{Z}_{2}(x):gr(p)<n \wedge g(x)|p(x)}$ por esto 
            se dice que $C$ es generador (son los multiplicos de $g(x)$ de menor grado).
        \item [2)] $C = \llaves{v(x)\odot g(x): v\in \mathbb{Z}_{2}(x)}$ son los multiplos 
            de $g$ modulares.
        \item [3)] Si $k=Dim(C)$ entonces $gr(g)=n-k$.
        \item [4)] $g(x) | (1+x^{n})$.
        \item [5)] Si $g(x) = g_{0} + g_{1}x + \ldots +$ entonces $g_{0}=1$.
    \end{itemize}
\end{teorema}

\underline{Prueba:}
\medskip

Sea $C_{1} = \llaves{p(x)\in \mathbb{Z}_{2}(x): gr(p)<n \wedge g(x)|p(x)}$ y
$C_{2} = \llaves{v(x)\odot g(x): v\in \mathbb{Z}_{2}(x)}$. Hay que demostrar que 
$C \subseteq C_{1}, C_{2} \subseteq C$ y $C_{1} \subseteq C_{2}$.
\medskip

Tenemos que ver $C_{1} \subseteq C_{2}$, entonces $p(x) \in C_{1}$ y veremos si 
esta en $C_{2}$.
\medskip

Entonces $gr(p)<n \wedge g(x)|p(x)$, entonces existe $q(x): p(x) = g(x)q(x)$
tomando modulo nos queda:
\begin{align*}
    p(x) \mod (1+x^{n}) &= g(x)q(x)\mod (1+x^{n})\\
    &= g(x) \odot q(x) \in C_{2}
\end{align*}

Como $gr(p) < n \Rightarrow p(x) \mod (1+x^{n}) = p(x) \Rightarrow p(x) \in C_{2}$.
\medskip

Ahora veamos que $C_{2} \subset C$, puesto que $g(x) \in C_{2}$ y debido a esto dada 
cualquier palabra $v(x)$ en C, $v(x) \odot g(x) \in C$
\medskip

$C \subseteq C_{1}$, Sea $p(x) \in C$ como las palabras de $C$ tienen longitud $n$, 
entonces $gr(p) < n$ (A).
\medskip

Dividamos $p(x)$ por $g(x)$ entonces existe $q(x)$ y $r(x)$ tal que:
\begin{align*}
    p(x) &= g(x)q(x) + r(x), \,\, gr(r) < gr(g)\\
    &\text{Tomando modulo y usando $gr(p)<n$}\\
    p(x) \mod (1+x^{n}) &= g(x)q(x) + r(x) \mod (1+x^{n})\\
    &= g(x)q(x) \mod (1+x^{n}) + \underbrace{r(x) \mod (1+x^{n})}_{gr(r)<gr(g)<n}\\
    p(x) &= g(x) \odot q(x) + r(x)
\end{align*}

Por lo tanto:
\begin{align*}
    r(x) = \underbrace{\underbrace{p(x)}_{\in C} + \underbrace{g(x) \odot q(x)}_{\in C_{2}}}_{\in C\,\,\text{porque $C$ es lineal}}
\end{align*}

Pero $gr(r) < gr(g) = \,\,\text{menor grado de un polinomio no nulo en $C$}$. 
Entonces $r(x) = 0 \Rightarrow p(x) = g(x)q(x) \Rightarrow g(x)|p(x)$ (B).
\medskip

Entonces por (A) y (B) $\Rightarrow C \subseteq C_{1}$
\medskip

\underline{Prueba de 3):} $C = C_{1}$ dice que:
\begin{align*}
    C &= \llaves{q(x)g(x): gr(qg)<n} \\
    &= \llaves{q(x)g(x): gr(q)+gr(g)<n} \\
    &= \llaves{q(x)g(x): gr(q)<n-gr(g)} \\
\end{align*}
Limito el grado de $q$. Entonces:
\begin{align*}
    |C| &= |\llaves{q(x)g(x):gr(q) < n-gr(g)}|\\
    &= \llaves{\text{Como conjunto no son iguales, pero tienen la misma $\#$ de elementos}}\\
    &= |\llaves{q(x):gr(q)<n-gr(g)}|\\
    &= 2^{n-gr(g)}\\
\end{align*}
Pero $|C| = 2^{k} \Rightarrow k = n- gr(g)$. Entonces $gr(g) = n-k$.
\medskip

\underline{4):} Dividamos $1+x^{n}$ por $g(x)$, entonces:
\begin{align*}
    1+x^{n} &= g(x)q(x) + r(x),\,\, gr(r) < gr(g)\\
    &=\llaves{\text{Despejando $r(x)$}}\\
    r(x) &= (1+x^{n}) + g(x)q(x)\\
    &=\llaves{\text{Como $gr(r) < gr(g) < n$ entonces $r(x) = r(x) \mod (1+x^{n})$}}\\
    r(x) &= ((1+x^{n}) + g(x)q(x)) \mod (1+x^{n})\\
    &= (1+x^{n}) \mod (1+x^{n}) + (g(x)q(x)) \mod (1+x^{n})\\
    &= 0 + q(x) \odot g(x)\\
    &= q(x) \odot g(x) \in C
\end{align*}
Como $r(x) \in C$ y $gr(r) < gr(g)$, entonces $r(x) = 0 \Rightarrow g(x)|(1+x^{n})$.

%%%%%% 15 %%%%%%
\begin{itemize}
    \item [15)] \textbf{Probar que $3$SAT es NP-completo}
    \label{dem:3sat}
\end{itemize}

\begin{teorema} (Karp, 1978) $3-$SAT es $NP-$COMPLETO.
\end{teorema}

Nosotros vamos a demostrar que $3-$Color es $NP-$COMPLETO, para ello usaremos 
$SAT \to 3-SAT \to 3-Color$, para el salto de $3-$SAT a $3-$Color usaremos 
un grafo bastante complejo que se puede colorear con $3$ colores. Cabe resaltar 
que el salto de $3$ a $2$ es $NP-$COMPLETO por esto.
\medskip

\underline{Prueba del teorema:}
\medskip

Veremos que $SAT \<_{p} 3-SAT$.
\medskip

Entonces, sea $B$ en forma conjuntiva normal ($CNF$) tal que 
$B = D_{1} \wedge \ldots \wedge D_{r}$ con 
$D_{j} = l_{1,j} \vee l_{2,j} \vee \ldots \vee l_{i,j}$ con $l_{r_{j},j}$ literales.
\medskip

Queremos construir $\widetilde{B}$ polinomialmente tal que $\widetilde{B}$ este en
$CNF$ con $3$ literales por disjunción tal que $B$ sea satisfacible si y solo si 
$\widetilde{B}$ es satisfacible.
\medskip

Voy a definir unos $E_{j}$ y $\widetilde{B} = E_{1} \wedge \ldots \wedge E_{n}$.
\medskip

Cada $E_{j}$ lo definimos a partir de $D_{j}$, y procedemos de diferente manera 
según la cantidad de literales $r_{j}$ presentes en $D_{j}$, entonces:
\medskip

Si $r_{j} = 3$ tenemos que $E_{j} = D_{j}$, por lo que no hago nada.
\medskip

Si $r_{j} < 3$ agregamos variables mudas que no afectan el resultado, vemos los casos:
\begin{itemize}
    \item $r_{j} = 2 \to D_{j}=l_{1,j}\vee l_{2,j}$\\
        $E_{j} = (l_{1,j} \vee l_{2,j} \vee y_{j}) \wedge (l_{1,j} \vee l_{2,j} \vee \bar{y}_{j})$
    \item $r=1 \to D_{j} = l_{1,j}$\\
        $E_{j} = (l_{1,j} \vee y_{1,j} \vee y_{2.j}) \wedge (l_{1,j} \vee \bar{y}_{1,j} \vee y_{2,j}) \wedge 
        (l_{1,j} \vee y_{1,j} \vee \bar{y}_{2,j}) \wedge (l_{1,j} \vee \bar{y}_{1,j} \vee \bar{y}_{2,j})$
\end{itemize}

Si $r_{j} > 3$ es el problema más grande:
$$D_{j} = l_{1,j} \vee l_{2,j} \vee \ldots \vee l_{r_{j},j}$$ 

entonces:
\begin{align*}
    E_{j} = &(l_{1,j} \vee l_{2,j} \vee y_{1,j}) \wedge (l_{3,j} \vee y_{2,j} \vee \bar{y}_{1,j}) \\
    &\wedge (l_{4,j} \vee y_{3,j} \vee \bar{y}_{2,j}) \wedge \ldots \wedge (l_{r_{j-2},j} \vee y_{r_{j-3},j} \vee \bar{y}_{r_{j-4},j})\\
    &\wedge (l_{r_{j-1},j} \vee l_{r_{j},j} \vee \bar{y}_{r_{j-3},j})
\end{align*}

Son variables booleanas para evaluarlas tengo que elegir un vector de $0's$ y $1's$.
\medskip

Ejemplo: Supongamos $\widetilde{B}$ satisfacible, $\widetilde{B}$ es suma de función de variables 
$x_{1},\ldots x_{algo}$ y variables extra $y_{i,j}$. Podriamos denotarlo con 
$\widetilde{B}(\overrightarrow{x}, \overrightarrow{y})$ mientras que $B$ depende 
solo de las $\overrightarrow{x}$, es decir, $B(\overrightarrow{x})$.
\medskip

Es decir:
$$\underbrace{B = D_{1} \wedge \ldots \wedge D_{n}}_{\overrightarrow{x}} \to \underbrace{\widetilde{B} = E_{1} \wedge \ldots \wedge E_{n}}_{\overrightarrow{x}, \overrightarrow{y}}$$

% Por lo que $\widetilde{B}$ satisfacible entonces existen vectores $\overrightarrow{a}, \overrightarrow{b}$
% tales que $\widetilde{B}(\overrightarrow{a}, \overrightarrow{b}) = 1$. 
Vamos a demostrar que si  
% $B(\overrightarrow{a}) = 1$.
$\exists \overrightarrow{a}: B(\overrightarrow{a}) = 1 \Leftrightarrow \exists \overrightarrow{c}, \overrightarrow{b} : \widetilde{B}(\overrightarrow{c}, \overrightarrow{b}) = 1$.
(la idea es ver que $\overrightarrow{a} = \overrightarrow{c}$)
\medskip

$(\Leftarrow)$ Supongamos que $\widetilde{B}(\overrightarrow{c}, \overrightarrow{b}) =1$ 
queremos probar $B(\overrightarrow{a}) = 1$. Supongamos que no se cumple y llegaremos 
a un absurdo, entonces se da $B(\overrightarrow{a}) = 0$.
\medskip

Como $B = D_{1} \wedge \ldots \wedge D_{n}$ entonces $\exists j : D_{j}(\overrightarrow{a}) = 0$, 
pero $D_{j} = l_{1,j} \vee l_{2,j} \vee \ldots \vee l_{r_{j},j}$ donde todos 
los terminos tienen que ser cero.
\medskip

Entonces, $l_{i,j}(\overrightarrow{a}) = 0\,\, \forall i$ (y ese $j$). Por otro 
lado $\widetilde{B}(\overrightarrow{a}, \overrightarrow{b}) = 1 \Rightarrow \exists E_{j}(\overrightarrow{a},\overrightarrow{b}) = 1\,\, \forall j$
en particular para ese $j$ que mencionamos antes.
\medskip

Ahora tenemos que ver los casos:
\begin{itemize}
    \item Si $r_{j} = 3$ tenemos que $E_{j} = D_{j}$ asi que esto es imposible.
        $D_{j}(\overrightarrow{a}) = 0 \Rightarrow E_{j} = 0 \Rightarrow \widetilde{B}(\overrightarrow{a},\overrightarrow{b}) = 0$
        Absurdo.
    \item Si $r_{j} = 2$ tenemos que 
        $E_{j} = (l_{1,j} \vee l_{2,j} \vee y_{j}) \wedge (l_{1,j} \vee l_{2,j} \vee \bar{y}_{j})$
        pero $l_{i,j}(\overrightarrow{a}) = 0$ entonces:
        \begin{align*}
            1 &= E_{j}(\overrightarrow{a},\overrightarrow{b}) \\
            &= (0 \vee 0 \vee y_{j}(\overrightarrow{b})) \wedge (0 \vee 0 \vee \bar{y}_{j}(\overrightarrow{b}))\\
            &= y_{j}(\overrightarrow{b}) \wedge \bar{y}_{j}(\overrightarrow{b})\\
            1 &= 0
        \end{align*}
        Absurdo.
    \item $r_{j} =1$
        \begin{align*}
            1 &= E_{j}(\overrightarrow{a},\overrightarrow{b}) \\
            &=(l_{1,j} \vee y_{1,j} \vee y_{2,j}) \wedge (l_{1,j} \vee y_{1,j} \vee \bar{y}_{2,j}) \wedge (l_{1,j} \vee \bar{y}_{1,j} \vee y_{2,j}) \wedge \\ 
            &(l_{1,j} \vee \bar{y}_{1,j} \vee \bar{y}_{2,j})[\overrightarrow{a},\overrightarrow{b}]\\
            &=(p \vee q) \wedge \underbrace{(\bar{p} \vee q) \wedge (p \vee \bar{q})}_{sii} \wedge (\bar{p} \vee \bar{q}) \\
            &=0
        \end{align*}
        Absurdo.
    \item $r_{j} > 4$
        \begin{align*}
            1 &= E_{j}(\overrightarrow{a},\overrightarrow{b}) \\
            &= (l_{1,j} \vee l_{2,j} \vee y_{1,j}) \wedge (l_{3,j} \vee \bar{y}_{1,j} \vee y_{2,j}) \wedge (l_{4,j} \vee \bar{y}_{2,j} \vee y_{3,j}) \wedge \ldots \\ 
            &\wedge (l_{r_{j}-2,j} \vee \bar{y}_{r_{j}-4,j} \vee y_{r_{j}-3,j}) \wedge (\bar{y}_{r_{j}-3,j} \vee l_{r_{j}-1,j} \vee l_{r_{j},j})[\overrightarrow{a},\overrightarrow{b}]\\
            & \text{sabemos que:}\,\, l_{i,j}(\overrightarrow{a}) = 0\\
            & \text{si}\,\, p_{i} = y_{i,j}(\overrightarrow{b}) = p_{1} \wedge (\bar{p}_{1} \vee p_{2}) \wedge (\bar{p}_{2} \vee p_{3}) \wedge \ldots \wedge (\bar{p}_{r_{j}-4} \vee p_{r_{j}-3}) \wedge \bar{p}_{r_{j}-3}\\
            & p_{1} \wedge (p_{1} \Rightarrow p_{2}) \wedge (p_{2} \Rightarrow p_{3}) \wedge \ldots \wedge (p_{r_{j}-4} \Rightarrow p_{r_{j}-3}) \wedge \bar{p}_{r_{j}-3} =0
        \end{align*}
        Todos tienen que ser $1$ y el último $0$, es un absurdo.
\end{itemize}

$(\Rightarrow)$ Si $\exists \overrightarrow{a} : B(\overrightarrow{a}) = 1 \Rightarrow \exists \overrightarrow{b} : \widetilde{B}(\overrightarrow{a},\overrightarrow{b}) = 1$
\medskip

Para $r_{j} \< 3$ se le puede dar cualquier valor a los $y_{i,j}$ por ejemplo $0$ 
y es trivial ver los casos $r_{j} = 1$ y $r_{j} = 2$
\medskip

El único problema grande es $r_{j} > 3$
\medskip

Como $B(\overrightarrow{a}) = 1$ entonces $D_{j}(\overrightarrow{a}) = 1\,\forall j$ 
(y al menos un término de $D_{j}$ es $1$). Entonces, 
$\exists i_{j} : l_{i_{j},j}(\overrightarrow{a}) = 1$.
Si hay más de uno tomo cualquiera, por ejemplo el primero.
\medskip

Evaluamos los $y_{i,j}$ de forma tal que:
\begin{align*}
    y_{i,j}(\overrightarrow{b}) = 1 &\,\, si,\,\, i = 1,\ldots i_{j}-2 \\
    y_{i,j}(\overrightarrow{b}) = 0 &\,\, si,\,\, i \> i_{j}-1
\end{align*}

Entonces, si evaluamos tenemos:
\begin{align*}
    E_{j}(\overrightarrow{a},\overrightarrow{b}) &= (l_{1,j} \vee l_{2,j} \vee \underbrace{y_{1,j}}_{=1}) \wedge \\
    & (l_{3,j} \vee \bar{y}_{1,j} \vee \underbrace{y_{2,j}}_{=1}) \wedge \ldots \wedge \\
    & (l_{i_{j}-1,j} \vee \bar{y}_{i_{j}-3,j} \vee \underbrace{y_{i_{j}-2,j}}_{=1}) \wedge \\
    & (\underbrace{l_{i_{j},j}}_{=1(t)} \vee \bar{y}_{i_{j}-2,j} \vee y_{i_{j}-1,j}) \wedge \\
    & (l_{i_{j}+1,j} \vee \underbrace{\bar{y}_{i_{j}-1},j}_{=1} \vee y_{i_{j},j}) \wedge \ldots \\
    % & (l_{i_{j}-1,j} \vee l_{i_{j},j} \vee \underbrace{\bar{y}_{i_{j}-3,j}}_{=1})\\
    &= 1
\end{align*}
$\square$

%%%%%% 16 %%%%%%
\begin{itemize}
    \item [16)] \textbf{Probar que $3-$COLOR es NP-completo}
    \label{dem:3color}
\end{itemize}

\begin{teorema} (Karp) $3-$Color es $NP-$COMPLETO.
\end{teorema}

\underline{Prueba:}
\medskip

Veremos que $3-$SAT $\leq_{p} 3-$Color. Es decir, dada una expresión booleana $B$ 
en $CNF$ con $3$ literales por disjunción, debemos construir polinomialmente un grafo 
$G$ tal que se cumpla que:
$$B \text{ es satisfacible} \Leftrightarrow \chi (G) \leq 3$$ 

Suponemos $B = D_{1} \wedge \ldots \wedge D_{m}$ con variables $x_{1}, \ldots x_{n}$ 
y $D_{j} = l_{1,j} \vee l_{2,j} \vee l_{3,j}$. (por ser $3-$SAT, esto queda fijo).
\medskip

Construcción polinomialmente del grafo $G$:
\medskip

Vertices:
$$\llaves{s,t} \cup \llaves{v_{l}: l\,\text{es literal}} \cup \llaves{a_{i,j},e_{i,j}}_{i= 1,2,3; j = 1,2,\ldots m}$$

donde $\llaves{v_{l}: l\,\text{es literal}}$ es equivalente a $\llaves{v_{x_{1}},v_{x_{2}},\ldots ,v_{x_{n}},v_{\bar{x}_{1}},v_{\bar{x}_{2}},\ldots,v_{\bar{x}_{n}}}$
\medskip

Lados:
\begin{itemize}
    \item $st$
    \item $tv_{l}$ para todo literal $l$
    \item $v_{x_{i}}v_{\bar{x}_{i}}\,\, \forall i=1\ldots,n$
    \item $a_{1,j}a_{2,j}$
    \item $a_{2,j}a_{3,j}$
    \item $a_{1,j}a_{3,j}$ estos últimos para $j = 1,2,\ldots,m$ y forman un 
        triangulo los tres.
    \item $a_{i,j}e_{i,j}$ para $i = 1,2,3$ y $j = 1,2,\ldots,m$
    \item $se_{i,j}$ para $i = 1,2,3$ y $j = 1,2,\ldots,m$
    \item Usando que es $3-$SAT, es decir, $D_{j} = l_{1,j} \vee l_{2,j} \vee l_{3,j}$ 
        entonces tenemos $e_{i,j}v_{l_{i},j}$ para $i = 1,2,3$ y $j = 1,2,\ldots,m$.
        Ejemplo: si $D_{j} = (x \vee \bar{x}_{2}\vee x_{4})$ entonces tengo 
        $\overrightarrow{v_{x}e_{1,j}}, \overrightarrow{v_{\bar{x}_{2}}e_{2,j}}, \overrightarrow{v_{x_{4}}e_{3,j}}$
\end{itemize}

El grafo es como en la siguiente imagen:
\begin{figure}[H]
    \centering
    \tikzset{every picture/.style={line width=0.75pt}} %set default line width to 0.75pt        
    \begin{tikzpicture}[x=0.75pt,y=0.75pt,yscale=-1,xscale=1]
    %uncomment if require: \path (0,345); %set diagram left start at 0, and has height of 345

    %Straight Lines [id:da024577145388274158] 
    \draw    (282,24) -- (181,101) ;
    %Straight Lines [id:da9147970884158108] 
    \draw    (282,24) -- (196,110) ;
    %Straight Lines [id:da24241664675971752] 
    \draw    (282,24) -- (233,127.86) ;
    %Straight Lines [id:da0019185310537932487] 
    \draw    (282,24) -- (248,133) ;
    %Straight Lines [id:da7044595354738321] 
    \draw    (282,24) -- (290,138) ;
    %Straight Lines [id:da12066461117319749] 
    \draw    (282,24) -- (308,138) ;
    %Straight Lines [id:da9216806050314463] 
    \draw    (282,24) -- (433,92) ;
    %Straight Lines [id:da8579836464252726] 
    \draw    (181,101) -- (196,110) ;
    %Straight Lines [id:da33046636333965984] 
    \draw    (233,127.86) -- (248,133) ;
    %Straight Lines [id:da8169841491168588] 
    \draw    (290,138) -- (308,138) ;
    %Shape: Triangle [id:dp5066112718925191] 
    \draw   (178,245) -- (213,285) -- (143,285) -- cycle ;
    %Straight Lines [id:da987499546806103] 
    \draw    (178,213) -- (178,245) ;
    %Straight Lines [id:da2327644054665976] 
    \draw    (212,215) -- (213,285) ;
    %Straight Lines [id:da6985649002842926] 
    \draw    (142,215) -- (143,285) ;
    %Shape: Triangle [id:dp9994407735338462] 
    \draw   (286,246) -- (321,286) -- (251,286) -- cycle ;
    %Straight Lines [id:da9924493799376253] 
    \draw    (286,214) -- (286,246) ;
    %Straight Lines [id:da27535775663210726] 
    \draw    (320,216) -- (321,286) ;
    %Straight Lines [id:da2242309455971152] 
    \draw    (250,216) -- (251,286) ;
    %Curve Lines [id:da11992808862863669] 
    \draw [color={rgb, 255:red, 74; green, 144; blue, 226 }  ,draw opacity=1 ]   (142,215) .. controls (152.54,207.09) and (201.5,171.7) .. (210,173) .. controls (218.5,174.3) and (403.54,114.09) .. (433,92) ;
    %Curve Lines [id:da7810373386432186] 
    \draw [color={rgb, 255:red, 74; green, 144; blue, 226 }  ,draw opacity=1 ]   (178,213) .. controls (188.54,205.09) and (237.5,169.7) .. (246,171) .. controls (254.5,172.3) and (403.54,114.09) .. (433,92) ;
    %Curve Lines [id:da7610424465229726] 
    \draw [color={rgb, 255:red, 74; green, 144; blue, 226 }  ,draw opacity=1 ]   (212,215) .. controls (222.54,207.09) and (271.5,171.7) .. (280,173) .. controls (288.5,174.3) and (403.54,114.09) .. (433,92) ;
    %Curve Lines [id:da4850711005271304] 
    \draw [color={rgb, 255:red, 74; green, 144; blue, 226 }  ,draw opacity=1 ]   (250,216) .. controls (260.54,208.09) and (309.5,172.7) .. (318,174) .. controls (326.5,175.3) and (403.54,114.09) .. (433,92) ;
    %Curve Lines [id:da8453334931293217] 
    \draw [color={rgb, 255:red, 74; green, 144; blue, 226 }  ,draw opacity=1 ]   (286,214) .. controls (296.54,206.09) and (345.5,170.7) .. (354,172) .. controls (362.5,173.3) and (403.54,114.09) .. (433,92) ;
    %Curve Lines [id:da5079195764877538] 
    \draw [color={rgb, 255:red, 74; green, 144; blue, 226 }  ,draw opacity=1 ]   (320,216) .. controls (330.54,208.09) and (379.5,172.7) .. (388,174) .. controls (396.5,175.3) and (403.54,114.09) .. (433,92) ;
    %Curve Lines [id:da8271646687449208] 
    \draw [color={rgb, 255:red, 74; green, 144; blue, 226 }  ,draw opacity=1 ]   (518.66,217.98) .. controls (510.9,209.95) and (476.64,173.98) .. (480.66,175.31) .. controls (484.68,176.63) and (453.68,115.45) .. (432,93) ;
    %Curve Lines [id:da6033929929298949] 
    \draw [color={rgb, 255:red, 74; green, 144; blue, 226 }  ,draw opacity=1 ]   (525.63,215.95) .. controls (517.87,207.92) and (483.61,171.95) .. (487.63,173.27) .. controls (491.65,174.6) and (453.68,115.45) .. (432,93) ;
    %Curve Lines [id:da7047394504171489] 
    \draw [color={rgb, 255:red, 74; green, 144; blue, 226 }  ,draw opacity=1 ]   (537.42,217.98) .. controls (529.66,209.95) and (495.41,173.98) .. (499.43,175.31) .. controls (503.45,176.63) and (453.68,115.45) .. (432,93) ;
    %Curve Lines [id:da6505082289688271] 
    \draw [color={rgb, 255:red, 74; green, 144; blue, 226 }  ,draw opacity=1 ]   (548.95,219) .. controls (541.19,210.97) and (506.93,175) .. (510.95,176.32) .. controls (514.97,177.65) and (453.68,115.45) .. (432,93) ;
    %Curve Lines [id:da3291128010874702] 
    \draw [color={rgb, 255:red, 74; green, 144; blue, 226 }  ,draw opacity=1 ]   (555.92,216.97) .. controls (548.16,208.93) and (513.9,172.97) .. (517.92,174.29) .. controls (521.94,175.61) and (453.68,115.45) .. (432,93) ;
    %Curve Lines [id:da1520010028888894] 
    \draw [color={rgb, 255:red, 74; green, 144; blue, 226 }  ,draw opacity=1 ]   (567.71,219) .. controls (559.95,210.97) and (525.69,175) .. (529.71,176.32) .. controls (533.73,177.65) and (453.68,115.45) .. (432,93) ;
    %Curve Lines [id:da7481436423375991] 
    \draw [color={rgb, 255:red, 65; green, 117; blue, 5 }  ,draw opacity=1 ]   (142,215) .. controls (182,185) and (156,140) .. (196,110) ;
    %Curve Lines [id:da00851059064414783] 
    \draw [color={rgb, 255:red, 65; green, 117; blue, 5 }  ,draw opacity=1 ]   (178,213) .. controls (218,183) and (193,157.86) .. (233,127.86) ;
    %Curve Lines [id:da7905929568459347] 
    \draw [color={rgb, 255:red, 65; green, 117; blue, 5 }  ,draw opacity=1 ]   (212,215) .. controls (252,185) and (231,170) .. (271,140) ;
    %Curve Lines [id:da5572121126151761] 
    \draw [color={rgb, 255:red, 65; green, 117; blue, 5 }  ,draw opacity=1 ]   (250,216) .. controls (290,186) and (141,131) .. (181,101) ;
    %Curve Lines [id:da9639491981576338] 
    \draw [color={rgb, 255:red, 65; green, 117; blue, 5 }  ,draw opacity=1 ]   (286,214) .. controls (326,184) and (208,163) .. (248,133) ;
    %Curve Lines [id:da32366690739652726] 
    \draw [color={rgb, 255:red, 65; green, 117; blue, 5 }  ,draw opacity=1 ]   (320,216) .. controls (360,186) and (239,173) .. (279,143) ;

    % Text Node
    \draw (436,76.8) node [anchor=north west][inner sep=0.75pt]    {$s$};
    % Text Node
    \draw (276,9.8) node [anchor=north west][inner sep=0.75pt]    {$t$};
    % Text Node
    \draw (164,94.8) node [anchor=north west][inner sep=0.75pt]    {$v_{x_{1}}$};
    % Text Node
    \draw (213,126.8) node [anchor=north west][inner sep=0.75pt]    {$v_{x_{2}}$};
    % Text Node
    \draw (184,114.8) node [anchor=north west][inner sep=0.75pt]    {$v_{\overline{x}_{1}}$};
    % Text Node
    \draw (242,136.8) node [anchor=north west][inner sep=0.75pt]    {$v_{\overline{x}_{2}}$};
    % Text Node
    \draw (265,123.8) node [anchor=north west][inner sep=0.75pt]    {$\dotsc $};
    % Text Node
    \draw (124,291.8) node [anchor=north west][inner sep=0.75pt]    {$a_{1,1}$};
    % Text Node
    \draw (203,291.8) node [anchor=north west][inner sep=0.75pt]    {$a_{3,1}$};
    % Text Node
    \draw (169,256.8) node [anchor=north west][inner sep=0.75pt]    {$a_{2,1}$};
    % Text Node
    \draw (242,291.8) node [anchor=north west][inner sep=0.75pt]    {$a_{1,2}$};
    % Text Node
    \draw (308,292.8) node [anchor=north west][inner sep=0.75pt]    {$a_{3,2}$};
    % Text Node
    \draw (276,257.8) node [anchor=north west][inner sep=0.75pt]    {$a_{2,2}$};
    % Text Node
    \draw (116,209.8) node [anchor=north west][inner sep=0.75pt]    {$e_{1,1}$};
    % Text Node
    \draw (153,210.8) node [anchor=north west][inner sep=0.75pt]    {$e_{2,1}$};
    % Text Node
    \draw (187,211.8) node [anchor=north west][inner sep=0.75pt]    {$e_{3,1}$};
    % Text Node
    \draw (255,211.8) node [anchor=north west][inner sep=0.75pt]    {$e_{1,2}$};
    % Text Node
    \draw (292,212.8) node [anchor=north west][inner sep=0.75pt]    {$e_{2,1}$};
    % Text Node
    \draw (325,210.8) node [anchor=north west][inner sep=0.75pt]    {$e_{3,2}$};
    % Text Node
    \draw (378,247.8) node [anchor=north west][inner sep=0.75pt]    {$\dotsc $};
    % Text Node
    \draw (69,269.8) node [anchor=north west][inner sep=0.75pt]    {$garras\ \rightarrow $};
    \end{tikzpicture}
\end{figure}

Observemos que $G$ tiene un triangulo, entonces $\chi(G) \> 3$, por lo que
$$\chi(G) \< 3 \Leftrightarrow \chi(G) = 3$$

Probemos primero:
$$\chi(G) = 3 \Rightarrow B \text{ es satisfacible}$$

Como $\chi (G) = 3$ entonces existe un coloreo propio de $G$ con $3$ colores, 
llamemosle $C$.
\medskip

Necesitamos definir un vector $(b_{1},b_{2},\ldots ,b_{n}) \in \llaves{0,1}^{n}$ 
tal que $B(\overrightarrow{b}) = 1$.
\medskip

Entonces, en base al coloreo $C$ definimos:
$$b_{i}= \left\{ \begin{array}{lcc}
    1 & \to & C(v_{x_{i}}) = C(s) \\
    \\ 0& \to & c.c\\
    \end{array}
    \right.$$

Queremos ver $B(\overrightarrow{b}) = 1$ basta ver que $D_{j}(\overrightarrow{b}) = 1\,\, \forall j$. 
Entonces tomemos un $j$ cualquiera, como para todo $j$ los $a_{1,j}a_{2,j}a_{3,j}$ 
forman un triangulo entonces los $3$ colores deben aparece ahí.
\medskip

En particular $\exists i: C(a_{i,j}) = C(t)$ (voy desde la garra hacia arriba). 
\medskip

Como $e_{i,j}a_{i,j} \in E \Rightarrow C(e_{i,j}) \neq C(t)$.
\medskip

Como $e_{i,j}s \in E \Rightarrow C(e_{i,j}) \neq C(s)$.
\medskip

Como $ts \in E \Rightarrow C(t) \neq C(s)$.
\medskip

Entonces $C(e_{i,j}) = \text{tercer color}$ (el color que es distinto del color de 
$s$ y $t$).
\medskip

Ahora ya sabemos que color tienen los extremos de las garras. Entonces, vemos que
$v_{l_{i},j}e_{i,j} \in E \Rightarrow C(v_{l_{i},j}) \neq C(e_{i,j}) = \text{tercer color}$.
$\Rightarrow C(v_{l_{i},j}) = C(t)$ o $C(s)$
\medskip

Pero $tv_{l_{i},j} \in E \Rightarrow C(v_{l_{i},j})\neq C(t)$ entonces no queda 
otra que $C(v_{l_{i},j}) = C(s)$ que es el candidato para cuando evalue en $b_{i}=1$.
\medskip

El $l_{i,j}$ es un literal por lo tanto es una variable o una negación. Veamos 
primero el caso que es una variable:
\medskip

Entonces $\exists k : l_{i,j} = x_{k}$ entonces $v_{l_{i},j} = v_{x_{k}}$ a su vez 
se da que $C(v_{l_{i},j}) = C(s) \Rightarrow C(v_{x_{k}}) = C(s) \Rightarrow b_{k} = 1$.
\medskip

Entonces $l_{i,j} = x_{k} \Rightarrow l_{i,j}(\overrightarrow{b}) = x_{k}(\overrightarrow{b}) = b_{k} = 1 \Rightarrow D_{j}(\overrightarrow{b}) = 1$.
\medskip

Supongamos ahora que es negación de una variable:
$\exists k: l_{i,j} = \bar{x}_{k} \Rightarrow l_{i,j}(\overrightarrow{b}) = \bar{x}_{k}(\overrightarrow{b}) = 1 - b_{k}$

Por $k: l_{i,j} = \bar{x}_{k}$ podemos decir que $C(s) = C(v_{l_{i},j}) = C(v_{\bar{x}_{k}})$
\medskip

Pero $v_{x_{k}}v_{\bar{x}_{k}} \in E \Rightarrow C(v_{x_{k}}) \neq C(v_{\bar{x}_{k}})$
\medskip

Como son distintos y $C(v_{\bar{x}_{k}}) = C(s)$ entonces $C(v_{x_{k}}) \neq C(s) \Rightarrow b_{k} = 0$
\medskip

Entonces $\bar{x}_{b}(\overrightarrow{b}) = 1-0 = 1$

$(\Leftarrow)$ 

$$B \text{ es satisfacible} \Rightarrow \chi(G) \< 3$$

Supongamos $B$ satisfacible $\Rightarrow \exists \overrightarrow{b} \in \llaves{0,1}^{n} : B(\overrightarrow{b}) = 1$ 
y hay que definir un coloreo a partir de esto. Entonces, debemos definir un coloreo 
$C$.
\medskip

Definimos $C(s) = 1$ y $C(t) =2$. Hay que analizar si el coloreo es propio en cada caso.
\medskip

Entonces el lado $st$ no crea problemas $(NCP)$.
\medskip

Para los $v_{l}$ definimos, $C(v_{l}) = l(\overrightarrow{b})$, en otras palabras:
\begin{align*}
    C(v_{x_{i}}) &= b_{i} \in (0,1) \\
    C(v_{\bar{x}_{i}}) &= 1 + b_{i} \in (0,1)
\end{align*}

Entonces hay que ver los casos:
$$\underbrace{v_{x_{i}}}_{0\,\,\text{ó}\,\,1} \underbrace{v_{\bar{x}_{i}}}_{1\,\,\text{ó}\,\,0} \Rightarrow NCP $$

$$\underbrace{v_{l}}_{0\,\,\text{ó}\,\,1} \underbrace{t}_{2} \Rightarrow NCP $$

Ahora como $B(\overrightarrow{b}) = 1$ (esto hay que usarlo si o si) entonces 
$D_{j}(\overrightarrow{b}) = 1\,\, \forall j$.
\medskip

$$\Rightarrow \forall j \exists i = i_{j} : l_{i_{j},j}(\overrightarrow{b}) = 1$$

Si hay más de uno elijo uno por ejemplo el primero.
\medskip

Coloreamos los $a_{i,j}$ de la siguiente forma (base de la garra)
\begin{align*}
    C(a_{i_{j},j}) &= 2 \\
    C(a_{i,j}) &= \text{le doy color $1$ a uno y $0$ al otro}
\end{align*}

Cómo los colores de las $a_{i,j}$ son $0,1,2$ entonces el triangulo $NCP$ y coloreamos 
los $e's$ de la siguiente forma:
$$C(e_{i,j})= \left\{ \begin{array}{lcc}
    2 & \to & i \neq i_{j} \\
    \\ 0& \to & i = i_{j}\\
    \end{array}
    \right.$$

Chequeamos los lados:
\begin{align*}
    &i \neq i_{j} \to \underbrace{a_{i,j}}_{0\,\,\text{ó}\,\,1}\underbrace{e_{i,j}}_{2} \Rightarrow NCP \\
    &i = i_{j} \to \underbrace{a_{i_{j},j}}_{2}\underbrace{e_{i_{j},j}}_{0} \Rightarrow NCP \\
    &\underbrace{s}_{1}\underbrace{e_{i,j}}_{0\,\,\text{ó}\,\,2} \Rightarrow NCP \\
    &\text{solo queda ver los}\,\, e_{i,j}v_{l_{i},j}\\
    &i \neq i_{j} \to \underbrace{e_{i,j}}_{2}\underbrace{v_{l_{i},j}}_{0\,\,\text{ó}\,\,1} \Rightarrow NCP \\
    &i = i_{j} \to \\
    &C(e_{i_{j},j}) = 0\\
    &C(v_{{l_{i_{j},j}}}) = l_{i_{j},j}(\overrightarrow{b}) = 1\\
    &\text{es igual a $1$ por la elección del $i_{j}$}\\
    &\Rightarrow \underbrace{e_{i,j}}_{0}\underbrace{v_{l_{i},j}}_{1} \Rightarrow NCP
\end{align*}
$\square$


\end{document}